%!TEX root = ../Thesis.tex
\section{Ziele des Projektes (Fabia Schmid)}
\fancyhead[R]{Ziele des Projektes}

%%%%%%%%%%%%%%%%%%
%Fabia
%%%%%%%%%%%%%%%%%%
Das Projekt „TEN Manger“ umfasst die Planung und Entwicklung einer Applikation zur Erstellung und Verwaltung von ToDos, Events und Notes. Diese Applikation stellt die Prüfungsleistung für das Modul „Projekte der Wirtschaftsinformatik“ dar.

Die Umsetzung erfolgt durch das Team „Angry Nerds“, welches sich aus Ruthild Gilles, Fabia Schmid, Jan Beilfuß, Yannick Rüttgers, Robin Menzel, Florian Rath, Joscha Nassenstein und Sertan Cetin zusammensetzt.

Der Zeitrahmen für die Realisierung des Projektes erstreckt sich vom 05.09.2017 bis zum 07.02.2018. Die Organisation, wie die Vereinbarung von Meetings und die konkrete Aufgabenverteilung erfolgen selbständig innerhalb des Teams.

Die Grundlage für das Projekt bilden sowohl Vorkenntnisse aus vorherigen Vorlesungen, die von den Teammitgliedern erworben wurden, sowie die Einführung in die Android Programmierung im Rahmen der Vorlesung "Projekte der Wirtschaftsinformatik“.

Ziel des Projektes ist die Entwicklung einer Applikation in der Programmiersprache Java. Die Applikation soll auf einem Tablet mit dem Betriebssystem Android laufen und die TEN Verwaltung ermöglichen. Die Verwaltung soll die Erstellung von ToDos, Events und Notes ermöglichen, sowie die Verwaltung dieser. Die Verwaltung soll aus dem Anzeigen, Filtern, Bearbeiten und Löschen bestehen.

Neben der Applikation soll im Rahmen des Projektes noch eine ausführliche Dokumentation angefertigt werden, welche den ganzen Projektverlauf du das Endergebnis dokumentiert und erklärt. Diese muss fristgerecht mit der Applikation abgegeben werden.

\newpage

Um das Ziel des Projektes zu erreichen wurden verschiedene Termine festgelegt, welche im Folgenden aufgelistet werden:

•	12.09.2018 – Abgabe Projekttagebuch

•	31.10.2018 – Abgabe Projekttagebuch

•	19.12.2018 – Abgabe Projekttagebuch

•	07.02.2019 – Abgabe Dokumentation per Mail

•	09.02.2019 – Vorstellung der Applikation und Abgabe der Applikation

Das Projekt kann nur als erfolgreich bezeichnet werden, wenn alle Termine fristgerecht erfüllt wurden.


