%!TEX root = ../Thesis.tex
\section{Anhang - Quelltexte}
\fancyhead[R]{Anhang - Quelltexte}
\label{instal}

\subsection{Unterkapitelüberschrift}

Unterkapitel

\subsubsection{Unterunterkapitelüberschrift}

Unterunterkapitel

\subsection{Kapitel mit Abbildung}

So kann man Abbildungen einfügen:

\begin{figure}[hbt]
\centering
\begin{minipage}[t]{1\textwidth} % Breite, z.B. 1\textwidth		
\caption{Abbildungsbeschriftung} % Überschrift
\includegraphics[width=1\textwidth]{img/fhdw}\\ % Pfad
\source{\url{http://dominique-fleury.com/?p=302}} % Quelle
\end{minipage}
\end{figure}

\subsection{Quellcode}

\texttt{Als ob wir}\\
\texttt{den Quellcode}\\
\texttt{einzelnd}\\
\texttt{so in diese Zeilen}\\
\texttt{eintragen!}\\

\begin{figure}[H]
\begin{quote}
\begin{texttt}
Ganz viel Quellcode:
public class Delete {
    /* Ruthild Gilles (30.11.2018)
     Class Delete contains methods to delete the given TEN object.
    */

    public static void deleteTEN(String tenID) {
        DatabaseRepository databaseRepository = new DatabaseRepository();
        databaseRepository.deleteTEN(tenID);
    }

    public static void deleteMultipleTENs(ArrayList<String> tenIDs) {
        DatabaseRepository databaseRepository = new DatabaseRepository();
        for (String tenID : tenIDs) {
            databaseRepository.deleteTEN(tenID);
        }
    }
}
\end{texttt}
\end{quote}
\end{figure}
