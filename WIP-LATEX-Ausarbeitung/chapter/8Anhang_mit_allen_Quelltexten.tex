%!TEX root = ../Thesis.tex
\section{Anhang - Quelltexte}
\fancyhead[R]{Anhang - Quelltexte}
\label{instal}

%%%%%%%%%%%%%%%%%%%%%%%%%%%%%%%%%%%%%%%%%%
%Titel und Name ändern. Pfad ändern (nur Klassenname).
%\lstinputlisting [caption=Titel (Vorname Nachname)]{code/Klassenname.java}
%Alternativ in die Mitte den Quellcode kopieren
\begin{figure}[H]
\begin{lstlisting}[caption=Titel (Vorname Nachname)]

\end{lstlisting}
\end{figure}
%%%%%%%%%%%%%%%%%%%%%%%%%%%%%%%%%%%%%%%%%%


\subsection{Activities}
	\subsubsection{Event}
		\paragraph{Data}
		\paragraph{GUI}
		\paragraph{Logic}
			\subparagraph{Listener}
		\paragraph{Reminder}
	\subsubsection{Note}
		\paragraph{Note}
			\subparagraph{Data}
			\subparagraph{Gui}
			\subparagraph{Logic}
		\paragraph{Notetags}
	\subsubsection{Todo}
\lstinputlisting [caption=CheckedChangeListener (Sertan Cetin)]{code/CheckedChangeListener.java}
\lstinputlisting [caption=ClickListener (Sertan Cetin)]{code/ClickListener.java}
\lstinputlisting [caption=Data (Sertan Cetin)]{code/Data.java}
\lstinputlisting [caption=DateChecker (Florian Rath)]{code/DateChecker.java}
\lstinputlisting [caption=DatePickerFragment (Florian Rath)]{code/DatePickerFragment.java}
\lstinputlisting [caption=Gui (Florian Rath)]{code/Gui.java}
\lstinputlisting [caption=Init (Sertan Cetin)]{code/Init.java}
\lstinputlisting [caption=MenuItemClickListener (Florian Rath)]{code/MenuItemClickListener.java}
\lstinputlisting [caption=TasksAdapter (Sertan Cetin)]{code/TasksAdapter.java}
\lstinputlisting [caption=TextWatcher (Florian Rath)]{code/TextWatcher.java}
\lstinputlisting [caption=TodoApplicationLogic (Florian Rath)]{code/TodoApplicationLogic.java}
\lstinputlisting [caption=TouchListener (Sertan Cetin)]{code/TouchListener.java}


%?
\subsection{Overview}
	\subsubsection{Event Fragment}
	\subsubsection{Header}
		\paragraph{Create Fragment}
		\paragraph{Delete Fragment}
		\paragraph{Search Fragment}
	\subsubsection{Image Fragment}
	\subsubsection{Note Fragment}
	\subsubsection{Overview Activity}
		\paragraph{Fragment Manager}
	\subsubsection{Super Classes}
	\subsubsection{Todo Fragments}
%?


\subsection{Data}
	\subsubsection{Models}
		\paragraph{TENs}	
\begin{figure}[H]
\begin{lstlisting}[caption=Todo (Joscha Nassenstein)]
@JsonIgnoreProperties(ignoreUnknown = true)
public class Todo extends TEN {
    private double progress;
    private String note;
    private Date startDate;
    private Date endDate;
    private ArrayList<Task> tasks;
    //Constructors
    public Todo(){
        super();
        this.note = "";
        this.startDate = new Date();
        this.endDate = new Date();
        this.tasks = new ArrayList<>();
        tasks.add(new Task());
    }
    public Todo(String title){
        super(title);
        this.startDate = new Date();
        this.endDate = new Date();
        this.tasks = new ArrayList<>();
        tasks.add(new Task());
    }
    public Todo(String title, String note){
        super(title);
        this.note = note;
        this.startDate = new Date();
        this.endDate = new Date();
        this.tasks = new ArrayList<>();
        tasks.add(new Task());
    }
    public Todo(String title, String note, ArrayList<Task> tasks){
        super(title);
        this.note = note;
        this.tasks = tasks;
        this.progress = calculateProgress();
        this.startDate = new Date();
        this.endDate = new Date();
    }
    public Todo(String title, String note, ArrayList<Task> tasks, Date endDate){
        super(title);
        this.note = note;
        this.tasks = tasks;
        this.progress = calculateProgress();
        this.startDate = new Date();
        this.endDate = endDate;
    }
    public Todo(String title, String note, ArrayList<Task> tasks, Date startDate, Date endDate){
        super(title);
        this.note = note;
        this.tasks = tasks;
        this.progress = calculateProgress();
        this.startDate = startDate;
        this.endDate = endDate;
    }
    //all Attributes for complete reconstruction
    public Todo(String title, String ID, int color, int accentColor, Date dateOfCreation, String note, ArrayList<Task> tasks, Date startDate, Date endDate){
        super(title, ID, color, accentColor, dateOfCreation);
        this.note = note;
        this.tasks = tasks;
        this.progress = calculateProgress();
        this.startDate = startDate;
        this.endDate = endDate;
    }
    /*@Override
    public boolean isFound(String pSearchString) {
        return super.isFound(pSearchString)?super.isFound(pSearchString):note.contains(pSearchString);
    }*/
    public Bundle getBundle(){
        Bundle bundle = super.getBundle();
        bundle.putString(BundleKeys.KEY_TODO_NOTE, note);
        boolean[] status = new boolean[tasks.size()];
        String[] description = new String[tasks.size()];
        int index = 0;
        for(Task task : tasks){
            description[index] = task.getDescription();
            status[index] = task.getStatus();
            index++;
        }
        bundle.putBooleanArray(BundleKeys.KEY_TODO_STATUS, status);
        bundle.putStringArray(BundleKeys.KEY_TODO_DESCRIPTION, description);
        return bundle;
    }
    //Getters and Setters
    public double getProgress(){return progress;}
    public String getNote(){return note;}
    public void setNote(String note){this.note = note;}
    public Date getStartDate(){return startDate;}
    public void setStartDate(Date startDate){this.startDate = startDate;}
    public Date getEndDate(){return endDate;}
    public void setEndDate(Date endDate){this.endDate = endDate;}
    public ArrayList<Task> getTasks() {return tasks;}
    public void setTasks(ArrayList<Task> tasks) {
        this.tasks = tasks;
        progress = calculateProgress();
    }
    public double calculateProgress(){
        int completed = 0;
        if(tasks == null)
            return 0;
        for(Task task:tasks){
            if(task.getStatus())
                completed++;
        }
        return (double)completed/tasks.size();
    }
}
\end{lstlisting}
\end{figure}

\begin{figure}[H]
\begin{lstlisting}[caption=Event (Joscha Nassenstein)]

@JsonIgnoreProperties(ignoreUnknown = true)
public class Event extends TEN {
    private Date time;
    private ArrayList<Date> reminder;
    private String address;
    private RecurringType recurringType;
    //Constructor
    //empty default
    public Event() {
        super();
        this.time = new Date();
        this.reminder = new ArrayList<>();
        this.address = "";
        this.recurringType = RecurringType.NONE;
    }
    //simple for usage
    public Event(String title, Date time, ArrayList<Date> reminder, String address) {
        super(title);
        this.time = time;
        this.reminder = reminder;
        this.address = address;
        this.recurringType = RecurringType.NONE;
    }
    //all Attributes for reconstruction of complete Object
    public Event(String title, String ID, int color, int accentColor, Date dateOfCreation, Date time, ArrayList<Date> reminder, String address, RecurringType recurringType) {
        super(title, ID, color, accentColor, dateOfCreation);
        this.time = time;
        this.reminder = reminder;
        this.address = address;
        this.recurringType = recurringType;
    }
    //Getter and Setter
    public Date getTime() {
        return time;
    }
    public void setTime(Date time) {
        this.time = time;
    }
    public ArrayList<Date> getReminder() {
        return reminder;
    }
    public void setReminder(ArrayList<Date> reminder) {
        this.reminder = reminder;
    }
    public String getAddress() {
        return address;
    }
    public void setAddress(String address) {
        this.address = address;
    }
    public RecurringType getRecurringType() {
        return recurringType;
    }
    public void setRecurringType(RecurringType recurringType) {
        this.recurringType = recurringType;
    }
    public Bundle getBundle() {
        Bundle bundle = super.getBundle();
        bundle.putLong(BundleKeys.KEY_EVENT_TIME, time.getTime());
        bundle.putString(BundleKeys.KEY_EVENT_ADDRESS, address);
        return bundle;
    }
}
\end{lstlisting}
\end{figure}

\begin{figure}[H]
\begin{lstlisting}[caption=Note (Joscha Nassenstein)]
@JsonIgnoreProperties(ignoreUnknown = true)
public class Note extends TEN {
    private String description;
    private ArrayList<String> tags;
    private ArrayList<Image> pictures;
    public int imageIDCounter;
    //Constructors
    public Note() {
        super();
        this.description = "";
        this.tags = new ArrayList<>();
        this.pictures = new ArrayList<>();
        this.imageIDCounter = 0;
    }
    public Note(String title, String description) {
        super(title);
        this.description = description;
        this.tags = new ArrayList<String>();
        this.pictures = new ArrayList<Image>();
    }
    //all Attributes for complete Reconstruction
    public Note(String title, String ID, int color, int accentColor, Date dateOfCreation, String description, ArrayList<String> tags, ArrayList<Image> pictures, int imageIDCounter) {
        super(title, ID, color, accentColor, dateOfCreation);
        this.description = description;
        this.tags = tags;
        this.pictures = pictures;
        this.imageIDCounter = imageIDCounter;
    }
    //Getter and Setter
    public String getDescription() {
        return description;
    }
    public void setDescription(String description) {
        this.description = description;
    }
    public ArrayList<String> getTags() {
        return tags;
    }
    public void setTags(ArrayList<String> tags) {
        this.tags = tags;
    }
    public Image addImage(Bitmap bitmap) {
        this.imageIDCounter++;
        String imageID = this.getID() + FileSystemConstants.IMAGE_CORE_ID + this.imageIDCounter;
        Image image = new Image(imageID, bitmap);
        Log.i("cool", image.getId());
        this.pictures.add(image);
        return image;
    }
    public void addImage(Image pImage) {
        Log.i("NoteRemake", "Pictures Size: " + pictures.size());
        for (int i = 0; i < pictures.size(); i++) {
            if (pImage.getId().equals(pictures.get(i).getId())) {
                pictures.set(i, pImage);
            }
        }
    }
    public ArrayList<Image> getPictures() {
        return pictures;
    }
    public Bundle getBundle() {
        Bundle bundle = super.getBundle();
        bundle.putString(BundleKeys.KEY_NOTE_DESCRIPTION, description);
        bundle.putStringArrayList(BundleKeys.KEY_NOTE_TAGS, tags);
        if (!pictures.isEmpty())
            bundle.putString(BundleKeys.KEY_NOTE_PICTURES, pictures.get(0).getId());
        return bundle;
    }
    public void imageNotFound(Image image) {
        for (int i = 0; i < this.getPictures().size(); i++) {
            if (image.getId().equals(this.getPictures().get(i).getId())) {
                this.getPictures().remove(i);
            }
        }
    }
    @Override
    public boolean isFound(String pSearchString){
        if(super.isFound(pSearchString))
            return true;
        for(String tag : tags) {
            if (tag.toLowerCase().contains(pSearchString.toLowerCase()))
                return true;
        }
        return false;
    }
}
\end{lstlisting}
\end{figure}

\begin{figure}[H]
\begin{lstlisting}[caption=TEN (Joscha Nassenstein)]
public class TEN {
    private String title;
    @JsonIgnore
    private String ID;
    @ColorInt
    @JsonIgnore
    private int color;
    @ColorInt
    @JsonIgnore
    private int accentColor;
    @JsonIgnore
    private Date dateOfCreation;
    //Constructor
    //empty default
    public TEN() {
        this.ID = null;
        this.title = "";
        int colorIndex = Colors.getRandomColorIndex();
        this.color = Colors.COLORS[colorIndex];
        this.accentColor = Colors.DARKER_ACCENT_COLORS[colorIndex];
        this.dateOfCreation = new Date();
    }
    //simple for usage
    public TEN(String title) {
        this.ID = null;
        this.title = title;
        int colorIndex = Colors.getRandomColorIndex();
        this.color = Colors.COLORS[colorIndex];
        this.accentColor = Colors.DARKER_ACCENT_COLORS[colorIndex];
        this.dateOfCreation = new Date();
    }
    //complete Object must be reconstructed
    public TEN(String title, String ID, int color, int accentColor, Date dateOfCreation) {
        this.ID = ID;
        this.title = title;
        this.color = color;
        this.accentColor = accentColor;
        this.dateOfCreation = dateOfCreation;
    }
    public boolean isFound(String pSearchString){
        return title!=null?title.toLowerCase().contains(pSearchString.toLowerCase()):false;
    }
    public Bundle getBundle() {
        Bundle bundle = new Bundle();
        bundle.putString(BundleKeys.KEY_TEN_ID, ID);
        bundle.putString(BundleKeys.KEY_TEN_TITLE, title);
        bundle.putInt(BundleKeys.KEY_TEN_COLOR, color);
        bundle.putInt(BundleKeys.KEY_TEN_ACCENT_COLOR, accentColor);
        bundle.putLong(BundleKeys.KEY_TEN_DATE_OF_CREATION, this.dateOfCreation.getTime());
        return bundle;
    }
    //Getter and Setter
    public String getTitle() {
        return title;
    }
    public void setTitle(String title) {
        this.title = title;
    }
    public String getID() {
        return ID;
    }
    public void setID(String ID) {
        this.ID = ID;
    }
    public int getColor() {
        return color;
    }
    public int getAccentColor() {
        return accentColor;
    }
    public void setColor(int color) {
        this.color = color;
    }
    public void setAccentColor(int accentColor) {
        this.accentColor = accentColor;
    }
    public void setDateOfCreation(Date dateOfCreation) {
        this.dateOfCreation = dateOfCreation;
    }
    public Date getDateOfCreation() {
        return dateOfCreation;
    }
}
\end{lstlisting}
\end{figure}
		\paragraph{Utils}
\begin{figure}[H]
\begin{lstlisting}[caption=BundleKeys (Jan Beilfuß)]
public class BundleKeys {
    //TEN Keys
    public static final String KEY_TEN_ID = "TenID";
    public static final String KEY_TEN_TITLE = "TenTitle";
    public static final String KEY_TEN_COLOR = "TenColor";
    public static final String KEY_TEN_ACCENT_COLOR = "TenAccentColor";
    public static final String KEY_TEN_DATE_OF_CREATION = "TenDateOfCreation";
    //Event Keys
    public static final String KEY_EVENT_TIME = "EventTime";
    public static final String KEY_EVENT_ADDRESS = "EventAdress";
    //Note Keys
    public static final String KEY_NOTE_DESCRIPTION = "NoteDescription";
    public static final String KEY_NOTE_TAGS = "NoteTags";
    public static final String KEY_NOTE_PICTURES = "NotePictures";
    //To Do Keys
    public static final String KEY_TODO_NOTE = "TodoNote";
    public static final String KEY_TODO_STATUS = "TodoStatus";
    public static final String KEY_TODO_DESCRIPTION = "TodoDescription";
}
\end{lstlisting}
\end{figure}

\begin{figure}[H]
\begin{lstlisting}[caption=Colors (Jan Beilfuß)]
public class Colors {
    //700er Material Colors
    public static final int[] COLORS = {
            0xFF0288D1, //light blue
            0xFFEF5350, //red
            0xFFAFB42B, //lime
            0xFF388E3C, //green
            0xFFFB8C00, //Orange
            0xFF7E57C2, //Deep Purple
            0xFFEC407A, //Pink
            0xFF009688, //Teal
    };
    //900er Material Colors
    public static final int[] DARKER_ACCENT_COLORS = {
            0xFF01579B, //light blue
            0xFFB71C1C, //red
            0xFF827717, //lime
            0xFF1B5E20, //green
            0xFFEF6C00, //orange
            0xFF512DA8, //Deep Purple
            0xFFC2185B, //Pink
            0xFF00796B, //Teal
    };
    public static int getRandomColorIndex() {
        int index = Colors.COLORS.length;
        while (index == Colors.COLORS.length) {
            index = (int) (Math.random() * Colors.COLORS.length);
        }
        return index;
    }
}
\end{lstlisting}
\end{figure}

\begin{figure}[H]
\begin{lstlisting}[caption=Image (Jan Beilfuß)]
public class Image {
    private String id;
    @JsonIgnore
    private Bitmap bitmap;
    //Constructor
    public Image() {
        this.id = "";
        this.bitmap = null;
    }
    public Image(String id) {
        this.id = id;
        this.bitmap = null;
    }
    public Image(String id, Bitmap bitmap) {
        this.id = id;
        this.bitmap = bitmap;
    }
    public Image (Image image){
        this.id = "" + image.getId();
        if(image.getBitmap()!=null){
            this.bitmap = Bitmap.createBitmap(image.getBitmap());
        }
    }
    //Getter and Setter
    public String getId() {
        return id;
    }
    public void setId(String id) {
        this.id = id;
    }
    public Bitmap getBitmap() {
        return bitmap;
    }
    public void setBitmap(Bitmap bitmap) {
        this.bitmap = bitmap;
    }
}
\end{lstlisting}
\end{figure}

\begin{figure}[H]
\begin{lstlisting}[caption=MockData (Jan Beilfuß)]

\end{lstlisting}
\end{figure}

\begin{figure}[H]
\begin{lstlisting}[caption=ReccuringType (Joscha Nassenstein)]
public enum RecurringType {
    NONE,DAILY,WEEKLY,MONTHLY,YEARLY
}
\end{lstlisting}
\end{figure}

\begin{figure}[H]
\begin{lstlisting}[caption=Tasks (Joscha Nassenstein)]
public class Task {
    private String description;
    private boolean status;
    //Constructors
    public Task(){
        this.description = "";
        this.status = false;
    }
    public Task(String description){
        this.description = description;
        this.status = false;
    }
    public Task(String description, boolean status){
        this.description = description;
        this.status = status;
    }
    //Getters and Setters
    public String getDescription(){return description;}
    public void setDescription(String description) {this.description = description;}

    public boolean getStatus(){return status;}
    public void setStatus(boolean status){this.status = status;}
}
\end{lstlisting}
\end{figure}
	\subsubsection{Repository}
\begin{figure}[H]
\begin{lstlisting}[caption=DatabaseRepository (Jan Beilfuß)]
/Class that manages All Requests to the Database
//Author: Jan Beilfuss
public class DatabaseRepository {
    ReadRepository mReadRepository;
    WriteRepository mWriteRepository;
    public DatabaseRepository() {
        this.mReadRepository = new ReadRepository();
        this.mWriteRepository = new WriteRepository();
    }
    public Todo getTodoByID(String pId) {
        return mReadRepository.getTodoByID(pId);
    }
    public Event getEventByID(String pId) {
        return mReadRepository.getEventByID(pId);
    }
    public Note getNoteByID(String pId) {
        return mReadRepository.getNoteByID(pId);
    }
    //for new TEN-Objects
    public void insertTEN(TEN pTen) { mWriteRepository.insertTEN(pTen); }
    //for already saved TEN-Objects
    public void updateTEN(TEN pTen) {
        mWriteRepository.updateTEN(pTen);
    }
    public ArrayList<TEN> getAllTENs() {
        return mReadRepository.getAllTENs();
    }
    public boolean deleteTEN(String pTenId) {
        return mWriteRepository.deleteTEN(pTenId);
    }
    //returns main and accent color for given TEN-Object-ID
    public int[] getTENColors(String pTenId) {
        return mReadRepository.getTENColors(pTenId);
    }
}
\end{lstlisting}
\end{figure}

\begin{figure}[H]
\begin{lstlisting}[caption=DataContextManager (Jan Beilfuß)]
//Class that provides the database to the DatabaseRepository-Classes and Functions
//Author: Jan Beilfuss
public class DataContextManager {
    public static Database database = null;
    public static Context context = null;
    //Gets Database from Context if it is not set
    public static void initDatabase(Context pContext) {
        if (DataContextManager.getDatabase() == null) {
            DataContextManager.context = null;
            DataContextManager.context = pContext;
            try {
                DatabaseConfiguration config = new DatabaseConfiguration(pContext.getApplicationContext());
                DataContextManager.database = new Database(RepositoryConstants.DATABASENAME, config);
                compactDatabase();
            } catch (CouchbaseLiteException e) {
                Toast.makeText(pContext, "Fehler bei der Datenbankerstellung", Toast.LENGTH_LONG);
            }
        }
    }
    //Compacts the Database = Cleans all Empty Entries (they are just flagged on Delete)
    public static void compactDatabase() {
        try {
            DataContextManager.getDatabase().compact();
        } catch (CouchbaseLiteException e) {
        }
    }
    //returns the number of Documents in the Database. ONLY needed when using Mockdata
    public static int getNumberOfDocuments() {
        Query query = QueryBuilder.select(SelectResult.expression(Meta.id)).from(DataSource.database(DataContextManager.getDatabase()));
        try {
            ResultSet rs = query.execute();
            return rs.allResults().size();
        } catch (CouchbaseLiteException e) {
            return -1;
        }
    }
    public static Database getDatabase() {
        return DataContextManager.database;
    }
}
\end{lstlisting}
\end{figure}

\begin{figure}[H]
\begin{lstlisting}[caption=RepositoryConstants (Jan Beilfuß)]
//Author: Jan Beilfuss
public class RepositoryConstants {
    //database name
    public static final String DATABASENAME = "TENDB";
    //Keys for Key-Value-Pairs in Document
    public static final String OBJECT_KEY = "ObjectKey";
    public static final String TYPE_KEY = "TypeKey";
    public static final String CREATION_DATE_KEY = "CreationDateKey";
    public static final String COLOR_KEY = "ColorKey";
    public static final String ACCENT_COLOR_KEY = "AccentColorKey";
    public static final String COUCHBASE_ID_KEY = "id";
    //Type corresponding to TEN-Classes
    public static final String EVENT_TYPE = "Event";
    public static final String NOTE_TYPE = "Note";
    public static final String TODO_TYPE = "Todo";
}
\end{lstlisting}
\end{figure}
		\paragraph{Converter}
\begin{figure}[H]
\begin{lstlisting}[caption=QuerriedTenConverter (Jan Beilfuß)]
//class that manages the Conversion from Query Result to TEN Java Object
//Author: Jan Beilfuss
public class QueriedTenConverter {

    private TenJsonParser mTenJsonParser;
    private SeparateAttributesConverter mSeparateAttributesConverter;

    public QueriedTenConverter() {
        this.mTenJsonParser = new TenJsonParser();
        this.mSeparateAttributesConverter = new SeparateAttributesConverter();
    }

    //Method that maps the Dictionary of a Query Result To a TEN-Object
    //Object only contains Information from the JSON, but non of the ones stored in the document
    public TEN createTENFromResult(Result pResult) {

        Dictionary dictionary = pResult.getDictionary(RepositoryConstants.DATABASENAME);
        String type = dictionary.getString(RepositoryConstants.TYPE_KEY);
        String objectJSON = dictionary.getString(RepositoryConstants.OBJECT_KEY);

        TEN resultTEN = null;
        switch (type) {
            case RepositoryConstants.EVENT_TYPE:
                resultTEN = mTenJsonParser.stringToEvent(objectJSON);
                break;
            case RepositoryConstants.NOTE_TYPE:
            resultTEN = mTenJsonParser.stringToNote(objectJSON);
                break;
            case RepositoryConstants.TODO_TYPE:
            resultTEN = mTenJsonParser.stringToTodo(objectJSON);
                break;
            default:
        }
        if (resultTEN != null) {
            resultTEN = this.mSeparateAttributesConverter.addTENPropertiesFromResult(resultTEN, pResult);
        }
        return resultTEN;
    }
}
\end{lstlisting}
\end{figure}

\begin{figure}[H]
\begin{lstlisting}[caption=SeparateAttributesConverter (Jan Beilfuß)]
//Methods that adds all Attributes not stored in the JSON
//Author: Jan Beilfuss
public class SeparateAttributesConverter {

    //Document is the Databasereturn when getting only a single TEN
    public TEN addTENPropertiesFromDocument(TEN pTen, Document pDocument) {
        pTen.setID(pDocument.getId());
        pTen.setColor(pDocument.getInt(RepositoryConstants.COLOR_KEY));
        pTen.setAccentColor(pDocument.getInt(RepositoryConstants.ACCENT_COLOR_KEY));
        pTen.setDateOfCreation(pDocument.getDate(RepositoryConstants.CREATION_DATE_KEY));
        return pTen;
    }

    //Result is the Databasereturn when querying a set of TENs
    public TEN addTENPropertiesFromResult(TEN pTen, Result pResult) {
        Dictionary dictionary = pResult.getDictionary(RepositoryConstants.DATABASENAME);
        pTen.setID(pResult.getString(RepositoryConstants.COUCHBASE_ID_KEY));
        pTen.setColor(dictionary.getInt(RepositoryConstants.COLOR_KEY));
        pTen.setAccentColor(dictionary.getInt(RepositoryConstants.ACCENT_COLOR_KEY));
        pTen.setDateOfCreation(dictionary.getDate(RepositoryConstants.CREATION_DATE_KEY));

        return pTen;
    }
}
\end{lstlisting}
\end{figure}

\begin{figure}[H]
\begin{lstlisting}[caption=TensJsonParser (Jan Beilfuß)]
//Class, that parses the JSON into a TEN Object
//Author: Jan Beilfuss
public class TenJsonParser {

    private ObjectMapper mObjectMapper;

    public TenJsonParser() {
        this.mObjectMapper = new ObjectMapper();
    }

    //parses To-do
    public Todo stringToTodo(String pJson) {
        try {
            Todo todo = this.mObjectMapper.readValue(pJson, Todo.class);
            return todo;
        } catch (IOException e) {
            return null;
        }
    }

    //parses Event
    public Event stringToEvent(String pJson) {
        try {
            Event event = this.mObjectMapper.readValue(pJson, Event.class);
            return event;
        } catch (IOException e) {
            return null;
        }
    }

    //parses Note
    public Note stringToNote(String pJson) {
        try {
            Note note = this.mObjectMapper.readValue(pJson, Note.class);
            return note;
        } catch (IOException e) {
            return null;
        }
    }
}
\end{lstlisting}
\end{figure}
		\paragraph{Filesystem}
\begin{figure}[H]
\begin{lstlisting}[caption=ImageDeleter (Jan Beilfuß)]
public class ImageDeleter {

    FileRepository mFileRepository;

    //Class that creates async Task to delete Images from the Filesystem
    //Author: Jan Beilfuss
    public ImageDeleter(FileRepository pFileRepository) {
        this.mFileRepository = pFileRepository;
    }

    public void execute(Image image) {
        DeleteImageTask deleteImageTask = new DeleteImageTask();
        deleteImageTask.execute(image);
    }

    private class DeleteImageTask extends AsyncTask<Image, Integer, Void> {

        //Deletes Image in preview and original folder
        @Override
        protected Void doInBackground(Image... images) {
            String[] folders = {FileSystemConstants.IMAGE_ORIGINAL_FOLDER, FileSystemConstants.IMAGE_PREVIEW_FOLDER};

            for (String folder : folders) {
                String directoryPath = mFileRepository.getContext().getExternalFilesDir(Environment.DIRECTORY_PICTURES).getAbsolutePath() + "/" + folder + "/";
                String filePath = directoryPath + images[0].getId() + ".jpg";

                File file = new File(filePath);
                if (file.exists()) {
                    file.delete();
                }
            }
            return null;
        }
    }
}
\end{lstlisting}
\end{figure}

\begin{figure}[H]
\begin{lstlisting}[caption=ImageSaver (Jan Beilfuß)]
public class ImageSaver {

    FileRepository mFileRepository;

    //Class that creates async Task to save Images to the filesystem
    //Author: Jan Beilfuss
    public ImageSaver(FileRepository pFileRepository) {
        this.mFileRepository = pFileRepository;
    }

    public void execute(Image image) throws IOException {
        Image copy = new Image(image);
        image.setBitmap(null);
        SaveImageTask saveImageTask = new SaveImageTask();
        saveImageTask.execute(copy);
    }

    private class SaveImageTask extends AsyncTask<Image, Integer, Void> {

        //saves Image to preview and original folder
        @Override
        protected Void doInBackground(Image... images) {
            PreviewImageCreator previewImageCreator = new PreviewImageCreator();
            Context context = DataContextManager.context;
            String[] folders = {FileSystemConstants.IMAGE_ORIGINAL_FOLDER, FileSystemConstants.IMAGE_PREVIEW_FOLDER};

            for (String folder : folders) {
                String folderPath = Environment.DIRECTORY_PICTURES + "/" + folder;
                String imageName = images[0].getId() + FileSystemConstants.JPEG;

                File storageDir = context.getExternalFilesDir(folderPath);
                FileOutputStream fos = null;
                Bitmap bitmap = images[0].getBitmap();
                File image = new File(storageDir, imageName);

                if (!image.exists()) {
                    if (bitmap != null) {
                        if (folder.equals(FileSystemConstants.IMAGE_PREVIEW_FOLDER)) {
                            bitmap = previewImageCreator.getPreviewImage(bitmap);
                        }
                        try {
                            fos = new FileOutputStream(image);
                        } catch (FileNotFoundException e) {
                        }

                        bitmap.compress(Bitmap.CompressFormat.JPEG, FileSystemConstants.COMPRESSION_FACTOR, fos);
                        try {
                            fos.flush();
                            fos.close();
                        } catch (IOException e) {
                        }
                    }
                }
            }
            images[0].setBitmap(null);
            return null;
        }
    }
}
\end{lstlisting}
\end{figure}

\begin{figure}[H]
\begin{lstlisting}[caption=FileRepositroy (Jan Beilfuß)]
//Class that handles everything regarding persistent Images
//Author: Jan Beilfuss
public class FileRepository {
    public Context getContext() {
        return DataContextManager.context;
    }

    //creates ImageSaver to save Images to the Filesystem
    public void saveImagePersistent(Image pImage) throws IOException {
        ImageSaver imageSaver = new ImageSaver(this);
        imageSaver.execute(pImage);
    }

    //reads and decodes Image from the Filesystem
    public Image readImageFromDirectory(Image image, String directory) {
        Context context = DataContextManager.context;
        String directoryPath = context.getExternalFilesDir(Environment.DIRECTORY_PICTURES).getAbsolutePath() + "/" + directory + "/";
        String filePath = directoryPath + image.getId() + FileSystemConstants.JPEG;
        Bitmap bitmap = BitmapFactory.decodeFile(filePath);
        Image result = new Image(image.getId(), bitmap);
        return result;
    }

    //deletes single file at given path
    public void deleteImageFromDirectory(String path) {
        File file = new File(path);
        if(file.exists()) file.delete();
    }

    //creates ImageDeleter to delete Files from orignal and preview folder
    public void deleteImageFromDirectories(Image image) {
        ImageDeleter imageDeleter = new ImageDeleter(this);
        imageDeleter.execute(image);
    }
}
\end{lstlisting}
\end{figure}

\begin{figure}[H]
\begin{lstlisting}[caption=FileSystemConstants (Jan Beilfuß)]
public class FileSystemConstants {
    public static final String JPEG = ".jpg";
    public static final int COMPRESSION_FACTOR = 75;

    public static final String IMAGE_CORE_ID = "imageID";

    public static final String IMAGE_ORIGINAL_FOLDER = "original";
    public static final String IMAGE_PREVIEW_FOLDER = "preview";
}
\end{lstlisting}
\end{figure}
		\paragraph{Sub-Repositories}
\begin{figure}[H]
\begin{lstlisting}[caption=GetAllTensQuery (Jan Beilfuß)]
//Class that contains our currently only Database-Query
//Author: Jan Beilfuss
public class GetAllTensQuery {

    private QueriedTenConverter mQueriedTenConverter;

    public GetAllTensQuery() { this.mQueriedTenConverter = new QueriedTenConverter(); }

    //Creates Query for all Saved TENs and returns converted java objects
    public ArrayList<TEN> getAllTENs() {
        Query query = QueryBuilder.select(
                SelectResult.all(),
                SelectResult.expression(Meta.id))
                .from(DataSource.database(DataContextManager.getDatabase()))
                .orderBy(Ordering.property(RepositoryConstants.CREATION_DATE_KEY).descending());
        ArrayList<TEN> resultList = new ArrayList<TEN>();

        try {
            ResultSet rs = query.execute();

            List<Result> allResults = rs.allResults();
            for (Result result : allResults) {
                TEN tenObject = this.mQueriedTenConverter.createTENFromResult(result);
                resultList.add(tenObject);
            }
            return resultList;

        } catch (CouchbaseLiteException e) {
            return null;
        }
    }
}
\end{lstlisting}
\end{figure}

\begin{figure}[H]
\begin{lstlisting}[caption=ReadRepository (Jan Beilfuß)]
//Class, that manages all reading Requests onto the Database
//Author: Jan Beilfuss
public class ReadRepository {
    TenJsonParser mTenJsonParser;
    SeparateAttributesConverter mSeparateAttributesConverter;
    GetAllTensQuery mGetAllTensQuery;

    public ReadRepository (){
        this.mTenJsonParser = new TenJsonParser();
        this.mSeparateAttributesConverter = new SeparateAttributesConverter();
        this.mGetAllTensQuery = new GetAllTensQuery();
    }

    //returns To-Do with given ID
    public Todo getTodoByID(String pId) {
        Document todoDocument = DataContextManager.getDatabase().getDocument(pId);
        String json = todoDocument.getString(RepositoryConstants.OBJECT_KEY);
        Todo finalTodo = this.mTenJsonParser.stringToTodo(json);
        finalTodo = (Todo) this.mSeparateAttributesConverter.addTENPropertiesFromDocument(finalTodo, todoDocument);
        return finalTodo;
    }

    //returns Event with given ID
    public Event getEventByID(String pId) {
        Document eventDocument = DataContextManager.getDatabase().getDocument(pId);
        String json = eventDocument.getString(RepositoryConstants.OBJECT_KEY);
        Event finalEvent = this.mTenJsonParser.stringToEvent(json);
        finalEvent = (Event) this.mSeparateAttributesConverter.addTENPropertiesFromDocument(finalEvent, eventDocument);
        return finalEvent;
    }

    //returns Note with given ID
    public Note getNoteByID(String pId) {
        Document noteDocument = DataContextManager.getDatabase().getDocument(pId);
        String json = noteDocument.getString(RepositoryConstants.OBJECT_KEY);
        Note finalNote = this.mTenJsonParser.stringToNote(json);
        finalNote = (Note) this.mSeparateAttributesConverter.addTENPropertiesFromDocument(finalNote, noteDocument);
        return finalNote;
    }

    //returns all saved TEN-Objects
    public ArrayList<TEN> getAllTENs() {
        return mGetAllTensQuery.getAllTENs();
    }

    //Needed when TEN is loaded asynchronously for setting the Color of the Activity
    public int[] getTENColors(String pTenId) {
        Document document = DataContextManager.getDatabase().getDocument(pTenId);
        int[] colors = new int[2];
        colors[0] = document.getInt(RepositoryConstants.COLOR_KEY);
        colors[1] = document.getInt(RepositoryConstants.ACCENT_COLOR_KEY);
        return colors;
    }
}
\end{lstlisting}
\end{figure}

\begin{figure}[H]
\begin{lstlisting}[caption=DocumentSaver (Jan Beilfuß)]
//class that manages the persistent saving process of Tens
//Author: Jan Beilfuss
public class DocumentSaver {

    ObjectMapper mObjectMapper;
    FileRepository mFileRepository;

    public DocumentSaver() {
        this.mObjectMapper = new ObjectMapper();
        this.mFileRepository = new FileRepository();
    }

    //manages process of writing a ten to a document and saving this
    public boolean updateCompleteDocument(TEN pTen, MutableDocument pMutableTENDocument) {
        pMutableTENDocument = setTypeOfDocument(pMutableTENDocument, pTen);
        pMutableTENDocument = saveSeparateAttributes(pMutableTENDocument, pTen);
        try {
            pMutableTENDocument.setString(RepositoryConstants.OBJECT_KEY, this.mObjectMapper.writeValueAsString(pTen));
            try {
                DataContextManager.getDatabase().save(pMutableTENDocument);
                return true;
            } catch (CouchbaseLiteException e) {
                return false;
            }
        } catch (JsonProcessingException e) {
            return false;
        }
    }

    //saves attributes not saved in the JSON-String
    private MutableDocument saveSeparateAttributes(MutableDocument pMutableDocument, TEN pTen){
        pMutableDocument.setDate(RepositoryConstants.CREATION_DATE_KEY, pTen.getDateOfCreation());
        pMutableDocument.setInt(RepositoryConstants.COLOR_KEY, pTen.getColor());
        pMutableDocument.setInt(RepositoryConstants.ACCENT_COLOR_KEY, pTen.getAccentColor());

        return pMutableDocument;
    }

    //Sets Type depending of the class of the given Object
    private MutableDocument setTypeOfDocument(MutableDocument pMutableDocument, TEN pTen) {

        if (pTen.getClass().getName().contains("Event")) {
            pMutableDocument.setString(RepositoryConstants.TYPE_KEY, RepositoryConstants.EVENT_TYPE);

        } else if (pTen.getClass().getName().contains("Note")) {
            pMutableDocument.setString(RepositoryConstants.TYPE_KEY, RepositoryConstants.NOTE_TYPE);

        } else if (pTen.getClass().getName().contains("Todo")) {
            pMutableDocument.setString(RepositoryConstants.TYPE_KEY, RepositoryConstants.TODO_TYPE);
        }
        return pMutableDocument;
    }
}

\end{lstlisting}
\end{figure}

\begin{figure}[H]
\begin{lstlisting}[caption=WriteRepository (Jan Beilfuß)]
//Class that handles all writing accesses to the database
//Author: Jan Beilfuss
public class WriteRepository {

    private DocumentSaver mDocumentSaver;
    private TenJsonParser mTenJsonParser;
    private FileRepository mFileRepository;

    public WriteRepository(){
        this.mDocumentSaver = new DocumentSaver();
        this.mTenJsonParser = new TenJsonParser();
        this.mFileRepository = new FileRepository();
    }

    //new TEN to the Database
    public void insertTEN(TEN pTen) {
        MutableDocument mutableTENDocument = new MutableDocument();
        pTen.setID(mutableTENDocument.getId());
        this.mDocumentSaver.updateCompleteDocument(pTen, mutableTENDocument);
    }

    //already Existing TEN to the Database
    public void updateTEN(TEN pTen) {
        MutableDocument mutableTENDocument = DataContextManager.getDatabase().getDocument(pTen.getID()).toMutable();
        this.mDocumentSaver.updateCompleteDocument(pTen, mutableTENDocument);
    }

    public boolean deleteTEN(String pTenId) {
        Document document = DataContextManager.getDatabase().getDocument(pTenId);
        deleteNoteImages(document);
        try {
            if(document != null){
                DataContextManager.getDatabase().delete(document);
                return true;
            }
            return false;
        } catch (CouchbaseLiteException e) {
            return false;
        }
    }

    //Side Effects of Deleting a Note
    public void deleteNoteImages(Document pDocument) {
        if (pDocument.getString(RepositoryConstants.TYPE_KEY).equals(RepositoryConstants.NOTE_TYPE)) {
            String json = pDocument.getString(RepositoryConstants.OBJECT_KEY);
            Note note = mTenJsonParser.stringToNote(json);
            for(Image image: note.getPictures()){
                mFileRepository.deleteImageFromDirectory(image);
            }
        }
    }
}
\end{lstlisting}
\end{figure}
	\subsubsection{Services}
\begin{figure}[H]
\begin{lstlisting}[caption=Create (Ruthild Gilles)]

public class Create {
    /* Ruthild Gilles
     Class Create contains methods to create new empty TEN objects.
     This class only exists to give a consistent form to the CRUD methods.
    */

    public static Todo newTodo() {
        return new Todo();
    }

    public static Event newEvent() {
        return new Event();
    }

    public static Note newNote() {
        return new Note();
    }
}
\end{lstlisting}
\end{figure}

\begin{figure}[H]
\begin{lstlisting}[caption=Read (Ruthild Gilles)]

public class Read {
     /* Ruthild Gilles
     Class Read contains methods to get all or one specific TEN object.
    */

    /*--------------------------------------------------
        Method to get all TEN objects in an arraylist
     --------------------------------------------------*/
    public static ArrayList<TEN> getAllTENs() {
        DatabaseRepository databaseRepository = new DatabaseRepository();
        ArrayList<TEN> allTEN;
        allTEN = databaseRepository.getAllTENs();
        Log.i("Mainfix", "Number Of TENs: " + allTEN.size());
        for (TEN ten : allTEN) {
            Log.i("Mainfix", "ID: " + ten.getID() + ", Titel: " + ten.getTitle());
        }
        return allTEN;
    }

    /*--------------------------------------------------
        Methods to get one TEN object by ID
     --------------------------------------------------*/
    public static Todo getTodoByID(String id) {
        DatabaseRepository databaseRepository = new DatabaseRepository();
        Todo todo = databaseRepository.getTodoByID(id);
        return todo;
    }

    public static Event getEventByID(String id) {
        DatabaseRepository databaseRepository = new DatabaseRepository();
        Event event = databaseRepository.getEventByID(id);
        return event;
    }

    public static Note getNoteByID(String id) {
        DatabaseRepository databaseRepository = new DatabaseRepository();
        Note note = databaseRepository.getNoteByID(id);
        return note;
    }

    public static int[] getColors(String tenID) {
        DatabaseRepository databaseRepository = new DatabaseRepository();
        int[] colors = databaseRepository.getTENColors(tenID);
        return colors;
    }
}
\end{lstlisting}
\end{figure}

\begin{figure}[H]
\begin{lstlisting}[caption=Update (Ruthild Gilles)]

public class Update {
    /* Ruthild Gilles
     Class Update contains methods to save information on a changed or newly created TEN.
    */

    /*--------------------------------------------------
        Methods for saving a TEN object
     --------------------------------------------------*/

    public static void saveTEN(TEN newTen) {
        DatabaseRepository databaseRepository = new DatabaseRepository();
        if (newTen.getID() == null) {
            databaseRepository.insertTEN(newTen);
        } else databaseRepository.updateTEN(newTen);
    }
}
\end{lstlisting}
\end{figure}

\begin{figure}[H]
\begin{lstlisting}[caption=Delete (Ruthild Gilles)]

public class Delete {
    /* Ruthild Gilles
     Class Delete contains methods to delete the given TEN object.
    */

    public static void deleteTEN(String tenID) {
        DatabaseRepository databaseRepository = new DatabaseRepository();
        databaseRepository.deleteTEN(tenID);
    }

    public static void deleteMultipleTENs(ArrayList<String> tenIDs) {
        DatabaseRepository databaseRepository = new DatabaseRepository();
        for (String tenID : tenIDs) {
            databaseRepository.deleteTEN(tenID);
        }
    }
}
\end{lstlisting}
\end{figure}

\begin{figure}[H]
\begin{lstlisting}[caption=ImageService (Ruthild Gilles)]

public class ImageService {

    public static void saveImage(Image image) {
        try {
            FileRepository fileRepository = new FileRepository();
            fileRepository.saveImagePersistent(image);
        } catch (IOException e) {
            Log.e("ImageService", e.getMessage());
        }
    }

    public static Image getImage(Image image) {
        FileRepository fileRepository = new FileRepository();
        Image result = fileRepository.readImageFromDirectory(image, FileSystemConstants.IMAGE_ORIGINAL_FOLDER);
        return result;
    }

    public static Image getPreviewImage(Image image) {
        FileRepository fileRepository = new FileRepository();
        Image result = fileRepository.readImageFromDirectory(image, FileSystemConstants.IMAGE_PREVIEW_FOLDER);
        return result;
    }

    public static void deleteImage(Image image) {
        FileRepository fileRepository = new FileRepository();
        fileRepository.deleteImageFromDirectories(image);
    }

    public static void deleteImage(String path) {
        FileRepository fileRepository = new FileRepository();
        fileRepository.deleteImageFromDirectory(path);
    }

    public static File createImageFile(Activity pActivity) throws IOException {
        String timeStamp = new SimpleDateFormat("yyyyMMdd_HHmmss", Locale.GERMANY).format(new Date());
        String imageFileName = "JPEG_" + timeStamp + "_";
        File storageDir = pActivity.getExternalFilesDir(Environment.DIRECTORY_PICTURES);
        return File.createTempFile(imageFileName, ".jpg", storageDir);
    }
}
\end{lstlisting}
\end{figure}

\subsection{Modules}
	\subsubsection{Image Compression}
\lstinputlisting [caption=ImageCompressionModule (Jan Beilfuß)]{code/ImageCompressionModule.java}
\lstinputlisting [caption=ImageRotationCorrectionModule (Jan Beilfuß)]{code/ImageRotationCorrectionModule.java}
\lstinputlisting [caption=ImageToolsModule (Jan Beilfuß)]{code/ImageToolsModule.java}
	\subsubsection{Share}
\lstinputlisting [caption=ShareModule (Jan Beilfuß)]{code/ShareModule.java}


