%!TEX root = ../Thesis.tex
\section{Anhang - Quelltexte}
\fancyhead[R]{Anhang - Quelltexte}
\label{instal}

\subsection{Activities}
	\subsubsection{Event}
		\paragraph{Data}
		\paragraph{GUI}
		\paragraph{Logic}
			\subparagraph{Listener}
		\paragraph{Reminder}
	\subsubsection{Note}
		\paragraph{Note}
			\subparagraph{Data}
			\subparagraph{Gui}
			\subparagraph{Logic}
		\paragraph{Notetags}
	\subsubsection{Todo}
\subsection{Data}
	\subsubsection{Models}
		\paragraph{TENs}
		\paragraph{Utils}
	\subsubsection{Repository}
		\paragraph{Converter}
		\paragraph{Filesystem}
		\paragraph{Sub-Repositories}
	\subsection{Services}
	\subsubsection{Tasks}
\subsection{Modules}
	\subsubsection{Image Compression}
	\subsubsection{Share}
\subsection{Overview}
	\subsubsection{Event Fragment}
	\subsubsection{Header}
		\paragraph{Create Fragment}
		\paragraph{Delete Fragment}
		\paragraph{Search Fragment}
	\subsubsection{Image Fragment}
	\subsubsection{Note Fragment}
	\subsubsection{Overview Activity}
		\paragraph{Fragment Manager}
	\subsubsection{Super Classes}
	\subsubsection{Todo Fragments}

%%%%%%%%%%%%%%%%%
%Beispiel
%%%%%%%%%%%%%%%%%
\begin{figure}[H]
\begin{lstlisting}[caption=Create Klasse (Ruthild Gilles)]

public class Create {
    /* Ruthild Gilles
     Class Create contains methods to create new empty TEN objects.
     This class only exists to give a consistent form to the CRUD methods.
    */

    public static Todo newTodo() {
        return new Todo();
    }

    public static Event newEvent() {
        return new Event();
    }

    public static Note newNote() {
        return new Note();
    }
}
\end{lstlisting}
\end{figure}

\begin{figure}[H]
\begin{lstlisting}[caption=Read Klasse (Ruthild Gilles)]

public class Read {
     /* Ruthild Gilles
     Class Read contains methods to get all or one specific TEN object.
    */

    /*--------------------------------------------------
        Method to get all TEN objects in an arraylist
     --------------------------------------------------*/
    public static ArrayList<TEN> getAllTENs() {
        DatabaseRepository databaseRepository = new DatabaseRepository();
        ArrayList<TEN> allTEN;
        allTEN = databaseRepository.getAllTENs();
        Log.i("Mainfix", "Number Of TENs: " + allTEN.size());
        for (TEN ten : allTEN) {
            Log.i("Mainfix", "ID: " + ten.getID() + ", Titel: " + ten.getTitle());
        }
        return allTEN;
    }

    /*--------------------------------------------------
        Methods to get one TEN object by ID
     --------------------------------------------------*/
    public static Todo getTodoByID(String id) {
        DatabaseRepository databaseRepository = new DatabaseRepository();
        Todo todo = databaseRepository.getTodoByID(id);
        return todo;
    }

    public static Event getEventByID(String id) {
        DatabaseRepository databaseRepository = new DatabaseRepository();
        Event event = databaseRepository.getEventByID(id);
        return event;
    }

    public static Note getNoteByID(String id) {
        DatabaseRepository databaseRepository = new DatabaseRepository();
        Note note = databaseRepository.getNoteByID(id);
        return note;
    }

    public static int[] getColors(String tenID) {
        DatabaseRepository databaseRepository = new DatabaseRepository();
        int[] colors = databaseRepository.getTENColors(tenID);
        return colors;
    }
}
\end{lstlisting}
\end{figure}

\begin{figure}[H]
\begin{lstlisting}[caption=Update Klasse (Ruthild Gilles)]

public class Update {
    /* Ruthild Gilles
     Class Update contains methods to save information on a changed or newly created TEN.
    */

    /*--------------------------------------------------
        Methods for saving a TEN object
     --------------------------------------------------*/

    public static void saveTEN(TEN newTen) {
        DatabaseRepository databaseRepository = new DatabaseRepository();
        if (newTen.getID() == null) {
            databaseRepository.insertTEN(newTen);
        } else databaseRepository.updateTEN(newTen);
    }
}
\end{lstlisting}
\end{figure}

\begin{figure}[H]
\begin{lstlisting}[caption=Delete Klasse (Ruthild Gilles)]

public class Delete {
    /* Ruthild Gilles
     Class Delete contains methods to delete the given TEN object.
    */

    public static void deleteTEN(String tenID) {
        DatabaseRepository databaseRepository = new DatabaseRepository();
        databaseRepository.deleteTEN(tenID);
    }

    public static void deleteMultipleTENs(ArrayList<String> tenIDs) {
        DatabaseRepository databaseRepository = new DatabaseRepository();
        for (String tenID : tenIDs) {
            databaseRepository.deleteTEN(tenID);
        }
    }
}
\end{lstlisting}
\end{figure}

