%!TEX root = ../Thesis.tex
\section{Fazit der Teammitglieder}
\fancyhead[R]{Fazit der Teammitglieder}
\label{instal}

\subsection{Fazit von Fabia Schmid}

Mein Fazit wird sich in ein generelles Fazit zu der Applikation, ein Fazit bezüglich der Gruppe und ein Fazit zu meiner eigenen Arbeit unterteilen.

Die Applikation, welche von uns, dem Team Angry Nerds, entwickelt wurde, entspricht den Vorgeben, die von dem Professor vorgegeben wurden und erweitert diese mit verschiedenen Funktionen. Zusätzlich ist die Oberfläche farbenfroh und benutzerfreundlich. Zusätzlich achteten wir auf eine einfache und intuitive Bedienung, wodurch auch eine hohe User-Akzeptanz erwartet wird.

Somit kann in Bezug auf die Applikation, das Fazit gezogen werden, dass die Applikation erfolgreich und sehr zufriedenstellend entwickelt wurde.

Auch das Fazit bezüglich der Gruppenarbeit fällt sehr positiv aus. Alle Gruppenmitglieder waren stets motiviert und haben sich an allen Schritten beteiligt. Bei Problemen probierten alle zu helfen und eine Lösung zu finden. Zusätzlich waren alle Teammitglieder sehr zuverlässig und zielorientiert. Durch die durchgängige Motivation war das ganze Team auf einer Wellenlänge und konnte gut zusammenarbeiten und das Projekt erfolgreich abschließen. Die Kommunikation wurde hier sehr vereinfacht, durch die Nutzung von Microsoft Teams und einer WhatsApp-Gruppe, in der schnell kleinere Fragen beantwortet werden konnten.

Die Projektsteuerung, welche meine Hauptaufgabe war, verlief auch erfolgreich. Die geplanten Aktivitäten konnten bis auf zwei Meilensteine fristgerecht erfüllt werden. Die beiden Meilensteine konnte ich jedoch durch verschieben trotzdem abschließen, wodurch das Projektziel nicht gefährdet wurde. Zusätzlich sorgte ich durch regelmäßige Erinnerungen für die Erfüllung der Aufgaben und koordinierte verschiedene Aufgaben im Team erfolgreich. Neben der Koordination, wozu auch die Zeitplanung des Projektes gehörte, erstellte ich in der Overview Activity alle Layouts und programmierte vier Klassen. Auch diese Aufgabe vollendete ich erfolgreich. Die Layouts passen zu den vorher festgelegten Vorgaben im Mockup und die Funktionen funktionieren wie geplant.

Somit kann auch in Bezug auf meine Arbeit, ein positives Fazit gezogen werden. Ich erfüllte die Aufgabe der Projektleiterin zufriedenstellend und beendete auch die programmatischen Aufgaben erfolgreich.

\subsection{Fazit von Florian Rath}

Die Entwicklung der TEN-App hat durch umfassende Projekttätigkeiten viele Aspekte unseres Studiums aufgegriffen und diese in dem praktischen Anteil, bei der Entwicklung der App, verdeutlicht. Durch dieses Projekt habe ich viele neue Erfahrungen gesammelt und habe einen tieferen Einblick in das Projektmanagement erhalten. Die praktische Anwendung hat mir dabei geholfen die Theorie besser zu verinnerlichen.

Während der Umsetzung sind mir eigene Planungsfehler aufgefallen, die ich in einem nächsten Projekt nicht mehr machen würde, z.B. die Aufteilung der Entwicklungspakete. Das Problem habe ich schon in dem Kapitel “Beschreibung von Problemen” genauer erläutert. Natürlich sind auch einige unvorhergesehene Probleme aufgetreten, die jedoch durch gute Teamarbeit bewältigt wurden. Die Kommunikation nahm dabei eine Schlüsselrolle ein und lief in unserer Gruppe, über Office Teams, Whatsapp und teilweise wurde Programmiersessions über Discord gehalten. Teams bietet die Möglichkeit sich in einem Chat auszutauschen und Dateien für jeden abzulegen, Whatsapp diente vor allem der schnellen und direkten Kommunikation. Discord ist eine Software für Sprachchats mit mehreren Leuten, in der wir uns bei Programmierproblemen austauschten. Die Tools wurden von unserer Gruppe gut genutzt und es fand ein produktiver Austausch statt. Jedoch haben gleichzeitige Diskussionen bei Teams und Whatsapp der Kommunikation ein wenig geschadet, allerdings lässt sich so etwas im Eifer des Gefechts nur schwer vermeiden. Diese Problematik ist auch eher selten aufgetreten. Ich finde Kommunikation in einem Team sehr wichtig und bin sehr zufrieden mit der Art und dem Ablauf der Angry Nerds. Die App wurde durch kontinuierliche Verbesserungsvorschläge auf ein immer höheres Niveau getrieben.

Leider lag der Großteil der Projektphase in einer ungünstigen Zeit. Es mussten viele Klausuren und Ausarbeitungen geschrieben. Zu dem kam unsere IHK-Abschlussprüfung hinzu. Das machte den zeitlichen Rahmen für das Projekt sehr eng. Trotzdem war es eine interessante Zeit, wenn auch fordernde Zeit, in der ich viel gelernt habe, z.B. den Umgang mit LaTeX, GitHub oder Android Studio. Abschließend lässt sich sagen, dass das Projekt erfolgreich abgeschlossen und fristgerecht fertiggestellt wurde.


\subsection{Fazit von Jan Beilfuß}

Fazit

\subsection{Fazit von Joscha Nassenstein}

Fazit

\subsection{Fazit von Robin Menzel}

Zu Begin der Gruppenarbeit war meine Freude über die Aufgabenstellung groß. Einen TEN-Manager kann ich gut gebrauchten - ein Kalorientagebuch, wie die Jahrgänge zuvor nicht. Bei der Größe der Gruppe hingegen war ich sehr gespannt, was das mit 8 Personen wird. Außerdem wollte ich mich schon immer mal mit der Android App entwicklung auseinander setzen. Aber direkt mit so einem großen Projekt? Was kann ich, fast ohne Vorkentnisse beitragen?

Unsere Gruppe, bzw. Teile davon haben sich bereits in anderen Projektarbeiten als gute Teams bewiesen. Alle kannten sich gut und konnten sich gegenseitig vom Kenntnissstand und von der Arbeitsmoral her gut einschätzen. So ware auch eine Aufgabenverteilung und eine Unterteilung in Sub-Teams sehr einfach umzusetzen. Und so konnte die Arbeit effektiv auf die Gruppe verteilt werden.

Der Start viel uns etwas schwer. Da es bei der Entwicklung nicht nur um das Coding geht, sondern zumindest mal feststehen muss, was überhaupt gecoded werden muss, wussten wir nicht, wie wird das am besten Aufteilen. Gerade bei Abhängikeiten zwischen Arbeitspaketen (Mock-Up erstellen \& Aufteilung der Activities) haben wir mehr Zeit benötigt, als notwendig. Dies hat sich in der Implementierung wiederholt, da die Abhängigkeit zwischen den Daten und den unterschiedlichen Activities hier sogar etwas höher ist. Durch guten und offenen Austausch und etwas flexibilität der Teammitglieder stellte dies aber keine große Herausforderung da.

Innerhalb der Gruppe fand ich den Wissensaustausch und die daraus enstehenden Synergieeffekte bemerkenswert. So war es uns über alle Activities hinweg möglich, mit jedem Wissensstand seinen Teil beizutragen. Themen wie das Sharing von TENs oder die Toolbar mussten nicht von jedem Sub-Team neu erfunden werden, sondern konnten direkt von Anfang an übernommen werden, wodurch mehr Zeit blieb weitere Herausforderungen zu meistern oder kleine extra Features einzubauen.

Für mich persönlich habe ich unglaublich viel über die App-Entwicklung gelernt und viel Spaß beim Finden von Lösungen gehabt. Meine Lernkurve war konstant stark ansteigend und die Erfahrung am Ende auch die Mühe wert.

Zusammenfassend ist mein Fazit sehr positiv. Mit dem Projektverlauf bin ich sehr zufrieden, wenn man bedenkt das von uns noch niemand ein so großes Software Engineering Projekt von Anfang bis Ende durchgeführt hat oder überhaupt mal eine App entwickelt hat. Die Applikation selbst hat meine Erwartungen übertroffen. Die Muss-Kriterien aus der Aufgabenstellung wurden alle mit etwas Interpretationsspielraum implementiert und viele kleine, nützlichen Extra Features haben es mit in die Applikation geschaft.

Lediglich bei der Dokumentation habe ich bei vielen Dingen nicht ganz den Sinn durchschaut. Wie eigentlich jeder Entwickler dokumentiere auch ich gerne nur das aller nötigste. Dennoch habe ich im Hinblick auf andere Dokumentation bei Software Engineering Projektenm, in dieser Dokumentation nicht überall einen Mehrwert entdeckt.

\subsection{Fazit von Ruthild Gilles}

Das Modul WIP enthält nicht nur die Programmierung einer Applikation, sondern für die erfolgreiche Implementierung wurde auch Projektmanagement benötigt. Diese Kombination finde ich realitätsnäher als es in anderen Modulen der Fall ist. Ein Projekt von vorne bis hinten zusammen in einem Team durchzuführen hat mir viele neue Erkenntnisse und Erfahrungen gebracht.

Zusätzlich wurden waren während des Projektes allerdings auch einige andere Fähigkeiten gefragt, welche ich noch nicht beherrschte. Dazu gehörte die Versionierung, welche wir mit Git Hub realisiert haben. Zu Beginn war es für mich eine Herausforderung die Funktionsweise von Git zu verstehen. Nach einiger Einarbeitungszeit beherrschte ich jedoch die Grundfunktionen, sodass ich diese Fähigkeit jetzt auch in meinem restlichen Leben einsetze kann. Abgesehen von der Versionierung, welche nicht für das Projekt gefordert war, stand mir die Aufgabe zu, mich mit dem Textverarbeitungsprogramm von Latex zu beschäftigen. Da uns hier ebenfalls jegliche Einweisung fehlte, benötigte ich auch hier zusätzliche Zeit zum Erlernen der benötigten Grundkenntnisse für Latex. Jedoch werde ich auch diese neuen Kenntnisse außerhalb von dem Modul einsetzten können.

In unserem Projektteam gab es einige sehr gute Entwickler und ich fand es eine Herausforderung mit dieser Leistung mitzuhalten. Wenn ich entwickle benötige ich deutlich mehr Zeit, um mich in die Logik hinein zu denken. Deswegen kam es dazu, dass die Implementierung von Logik von anderen Teammitgliedern übernommen wurden, da es für diese ein höherer Zeitaufwand gewesen wäre, mir die Logik der umgebenden Klassen und Methoden zu erklären, als es eben selbst schnell zu machen.

Ich war dem Datenteam zugeordnet und zu Beginn hatten wir Schwierigkeiten mit der Planung der persistenten Datenhaltung. Die Schnittstelle zwischen den Activities und den Datenbank-Klassen, die von mir entwickelt wurden, konnten wir zu Beginn nicht im Detail planen, da wir nicht genau wussten, welche Anforderungen die Activities an die Datenbank stellen würden. Dies lag an einer nicht ausgereiften Kommunikation zu Beginn des Projektes. Aus diesem Grund erstellte ich Klassen und musste diese später ändern und anpassen. Es war mehr ein Ausprobieren als ein strukturiert geplantes Entwickeln. Bei einem nächsten Projekt würde ich besonders zu Beginn, häufiger Teammeetings einplanen um gemeinsam das Grundgerüst der Applikation zu planen.

Zusätzlich habe ich den Zeitaufwand für die Fertigstellung der Ausarbeitung in Latex unterschätzt. Besonders da im Zeitraum von November bis Februar ausgesprochen viele Prüfungsleistungen und auch der Abschluss unserer Ausbildung anstanden, war es für alle Teammitglieder eine Herausforderung ihren Teil der Ausarbeitung bis zur Deadline fertig zu stellen. Letztendlich hat es jedoch noch alles geklappt.


\subsection{Fazit von Sertan Cetin}

Das Projekt hatte als Ziel, eine App innerhalb einer so großen Gruppe zu entwickeln. Dies war eine völlig neue Erfahrung für mich. Auch, wenn ich in der Programmierung allgemein etwas erfahren bin, war dieses Projekt eine große Bereicherung für mich, weil jeder etwas programmieren sollte und sich die App aus mehreren Einzelteilen zu einem Ganzen zusammensetzte. Bisher hatte ich nämlich immer im Alleingang irgendwelche Software entwickelt.

Auch wenn wir in der Berufsschule bereits in Gruppen etwas programmiert haben, ist es nicht vergleichbar. Dort haben wir mehr oder weniger immer am selben PC zusammengearbeitet und die Gruppen waren auch mit zwei bis drei Personen auch eher überschaubar. Da wir während des TEN-Projektes so viele waren, mussten wir z.B. GIT verwenden. Hier konnte ich von den Vorerfahrungen einiger aus der Gruppe stark profitieren. Ich lernte von ihnen, wofür Branches gut sind und wie man allgemein ein GIT-Repository am besten verwalten sollte.

Dass wir so viele Gruppenmitglieder waren, hatte natürlich auch Auswirkung auf die Kommunikation. Wir nutzten als Microsoft Teams und parallel dazu WhatsApp. Ich empfand die Kommunikation eher als schwierig. Wenn man mal einige Stunden offline war und genau dann in der WhatsApp-Gruppe viel diskutiert wurde, hatte ich mehrere hundert ungelesene Nachrichten in der Gruppe. Die Stärke von Teams war, dass man benachrichtigt wurde, wenn man in einem Chat erwähnt wurde. So konnte ich wichtige Informationen immer mitbekommen. Die Ergebnisse der WhatsApp-Diskussionen waren zumindest auch in Teams.

Zu Beginn des Projektes hatte ich mich geärgert, dass uns ein API-Level vorgegeben wurde. Neuere API-Level hätten die Programmierung von vielen Dingen sehr viel einfacher gemacht. Aber rückblickend bin ich froh, weil man gelernt hat, wie manche Mechanismen im Hintergrund passieren.

Ich fand es gut, dass wir für das gesamte Projekt mehrere Monate Zeit hatten. Auch wenn in diese Zeit sehr viele andere Verpflichtungen fielen, konnte man dies durch eine gute Planung kompensieren.

Als Abschluss lässt sich sagen, dass ich durch das Projekt so viel gelernt habe, dass wir als Team durch unser neues Wissen für ein ähnliches Projekt mit dem gleichen Umfang insgesamt viel weniger Zeit brauchen würden. Vor allem, weil man die meisten Klassen erneut verwenden könnte.

\subsection{Fazit von Yannick Rüttgers}

Das Projekt verlief für mich sehr zufriedenstellend. In diesem Fazit möchte ich auf die drei Punkte Applikation, Team und meine persönliche Entwicklung eingehen.

Mit dem Endergebnis des Projektes, der TEN-Manager App, bin ich sehr zufrieden. Mir gefällt sowohl das Design als auch die Nutzbarkeit des Produkts. Auch der Code, der hinter der Applikation steckt, ist ordentlich strukturiert und nutzt die Paradigmen der Objektorientierung sehr gut.

Die Arbeit im Team Angry Nerds hat mir größtenteils Freude bereitet. Bis auf einzelne Konflikte, die allerdings immer schnell gelöst werden können, funktionierte die Zusammenarbeit im Team sehr gut. Es wurde sich an die meisten Absprachen gehalten, und auf die Ergebnisse der anderen Projektbeteiligten konnte sich verlassen werden.

Meine persönliche Entwicklung würde ich auch als positiv beurteilen. Neben dem Erlernen der Programmierung von Android-Apps, habe ich mich sehr auf das Nutzen der Paradigmen der Objektorientierung, maßgeblich Polymorphie und Vererbung konzentriert. Der so entwickelte Programmcode wirkt sehr aufgeräumt, und bereits vorhandene Programmteile konnten oft wiederverwendet werden.



