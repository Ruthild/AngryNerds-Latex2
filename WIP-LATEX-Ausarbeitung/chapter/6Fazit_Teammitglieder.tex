%!TEX root = ../Thesis.tex
\section{Fazit der Teammitglieder}
\fancyhead[R]{Fazit der Teammitglieder}
\label{instal}

\subsection{Fazit von Fabia Schmid}

Mein Fazit wird sich in ein generelles Fazit zu der Applikation, ein Fazit bezüglich der Gruppe und ein Fazit zu meiner eigenen Arbeit unterteilen.

Die Applikation, welche von uns, dem Team Angry Nerds, entwickelt wurde, entspricht den Vorgeben, die von dem Professor vorgegeben wurden und erweitert diese mit verschiedenen Funktionen. Zusätzlich ist die Oberfläche farbenfroh und benutzerfreundlich. Zusätzlich achteten wir auf eine einfache und intuitive Bedienung, wodurch auch eine hohe User-Akzeptanz erwartet wird.

Somit kann in Bezug auf die Applikation, das Fazit gezogen werden, dass die Applikation erfolgreich und sehr zufriedenstellend entwickelt wurde.

Auch das Fazit bezüglich der Gruppenarbeit fällt sehr positiv aus. Alle Gruppenmitglieder waren stets motiviert und haben sich an allen Schritten beteiligt. Bei Problemen probierten alle zu helfen und eine Lösung zu finden. Zusätzlich waren alle Teammitglieder sehr zuverlässig und zielorientiert. Durch die durchgängige Motivation war das ganze Team auf einer Wellenlänge und konnte gut zusammenarbeiten und das Projekt erfolgreich abschließen. Die Kommunikation wurde hier sehr vereinfacht, durch die Nutzung von Microsoft Teams und einer WhatsApp-Gruppe, in der schnell kleinere Fragen beantwortet werden konnten.

Die Projektsteuerung, welche meine Hauptaufgabe war, verlief auch erfolgreich. Die geplanten Aktivitäten konnten bis auf zwei Meilensteine fristgerecht erfüllt werden. Die beiden Meilensteine konnte ich jedoch durch verschieben trotzdem abschließen, wodurch das Projektziel nicht gefährdet wurde. Zusätzlich sorgte ich durch regelmäßige Erinnerungen für die Erfüllung der Aufgaben und koordinierte verschiedene Aufgaben im Team erfolgreich. Neben der Koordination, wozu auch die Zeitplanung des Projektes gehörte, erstellte ich in der Overview Activity alle Layouts und programmierte drei Klassen. Auch diese Aufgabe vollendete ich erfolgreich. Die Layouts passen zu den vorher festgelegten Vorgaben im Mockup und die Funktionen funktionieren wie geplant.

Somit kann auch in Bezug auf meine Arbeit, ein positives Fazit gezogen werden. Ich erfüllte die Aufgabe der Projektleiterin zufriedenstellend und beendete auch die programmatischen Aufgaben erfolgreich.

\subsection{Fazit von Florian Rath}

Fazit

\subsection{Fazit von Jan Beilfuß}

Fazit

\subsection{Fazit von Joscha Nassenstein}

Fazit

\subsection{Fazit von Robin Menzel}

Fazit

\subsection{Fazit von Ruthild Gilles}

Das Modul WIP enthält nicht nur die Programmierung einer Applikation, sondern für die erfolgreiche Implementierung wurde auch Projektmanagement benötigt. Diese Kombination finde ich realitätsnäher als es in anderen Modulen der Fall ist. Ein Projekt von vorne bis hinten zusammen in einem Team durchzuführen hat mir viele neue Erkenntnisse und Erfahrungen gebracht.

Zusätzlich wurden waren während des Projektes allerdings auch einige andere Fähigkeiten gefragt, welche ich noch nicht beherrschte. Dazu gehörte die Versionierung, welche wir mit Git Hub realisiert haben. Zu Beginn war es für mich eine Herausforderung die Funktionsweise von Git zu verstehen. Nach einiger Einarbeitungszeit beherrschte ich jedoch die Grundfunktionen, sodass ich diese Fähigkeit jetzt auch in meinem restlichen Leben einsetze kann. Abgesehen von der Versionierung, welche nicht für das Projekt gefordert war, stand mir die Aufgabe zu, mich mit dem Textverarbeitungsprogramm von Latex zu beschäftigen. Da uns hier ebenfalls jegliche Einweisung fehlte, benötigte ich auch hier zusätzliche Zeit zum Erlernen der benötigten Grundkenntnisse für Latex. Jedoch werde ich auch diese neuen Kenntnisse außerhalb von dem Modul einsetzten können.

In unserem Projektteam gab es einige sehr gute Entwickler und ich fand es eine Herausforderung mit dieser Leistung mitzuhalten. Wenn ich entwickle benötige ich deutlich mehr Zeit, um mich in die Logik hinein zu denken. Deswegen kam es dazu, dass die Implementierung von Logik von anderen Teammitgliedern übernommen wurden, da es für diese ein höherer Zeitaufwand gewesen wäre, mir die Logik der umgebenden Klassen und Methoden zu erklären, als es eben selbst schnell zu machen.

Ich war dem Datenteam zugeordnet und zu Beginn hatten wir Schwierigkeiten mit der Planung der persistenten Datenhaltung. Die Schnittstelle zwischen den Activities und den Datenbank-Klassen, die von mir entwickelt wurden, konnten wir zu Beginn nicht im Detail planen, da wir nicht genau wussten, welche Anforderungen die Activities an die Datenbank stellen würden. Dies lag an einer nicht ausgereiften Kommunikation zu Beginn des Projektes. Aus diesem Grund erstellte ich Klassen und musste diese später ändern und anpassen. Es war mehr ein Ausprobieren als ein strukturiert geplantes Entwickeln. Bei einem nächsten Projekt würde ich besonders zu Beginn, häufiger Teammeetings einplanen um gemeinsam das Grundgerüst der Applikation zu planen.

Zusätzlich habe ich den Zeitaufwand für die Fertigstellung der Ausarbeitung in Latex unterschätzt. Besonders da im Zeitraum von November bis Februar ausgesprochen viele Prüfungsleistungen und auch der Abschluss unserer Ausbildung anstanden, war es für alle Teammitglieder eine Herausforderung ihren Teil der Ausarbeitung bis zur Deadline fertig zu stellen. Letztendlich hat es jedoch noch alles geklappt.


\subsection{Fazit von Sertan Cetin}

Fazit

\subsection{Fazit von Yannick Rüttgers}

Fazit


