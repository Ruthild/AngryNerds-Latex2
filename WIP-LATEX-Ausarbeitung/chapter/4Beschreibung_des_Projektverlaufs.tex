%!TEX root = ../Thesis.tex
\section{Beschreibung des Projektverlaufs}
\fancyhead[R]{Beschreibung des Projektverlaufs}
\label{instal}

\subsection{Tatsächliche Aufgabenverteilung im Team (Fabia Schmid)}
%%%%%%%%%%%
%Fabia
%%%%%%%%%%%

\begin{figure}[H]
\centering
\begin{minipage}[t]{1\textwidth} % Breite, z.B. 1\textwidth		
\caption{Tatsächliche Aufgabenverteilung} % Überschrift
\includegraphics[width=1\textwidth]{img/fhdw}\\ % Pfad
\source{Erstellt von Fabia Schmid} % Quelle
\end{minipage}
\end{figure}

Hier den Text einfach hin kopieren.

\subsection{Teammeetingprotokolle}
%%%%%%%%%%%
%Ruthild
%%%%%%%%%%%

Hier den Text einfach hin kopieren.

\subsection{Projekttagebücher aller Teammitglieder}
%%%%%%%%%%%
%Ruthild
%%%%%%%%%%%
\subsubsection{Projekttagebuch Jan Beilfuß}
\begin{longtable}{|p{10cm}|p{2cm}|p{2cm}|}
\hline
{\textbf{Beschreibung}} & {\textbf{Dauer}} & {\textbf{Datum}} \\ \hline

\end{longtable}

\newpage
\subsubsection{Projekttagebuch Joscha Nassenstein}
\begin{longtable}{|p{10cm}|p{2cm}|p{2cm}|}
\hline
{\textbf{Beschreibung}} & {\textbf{Dauer}} & {\textbf{Datum}} \\ \hline

\end{longtable}

\newpage
\subsubsection{Projekttagebuch Fabia Schmid}
\begin{longtable}{|p{10cm}|p{2cm}|p{2cm}|}
\hline
{\textbf{Beschreibung}} & {\textbf{Dauer}} & {\textbf{Datum}} \\ \hline
Aufgabe lesen und verstehen & 30 min & 05.09.2018 \\ \hline
Kick-Off Meeting zur Besprechung der Aufgabe, Verteilung der Rollen & 50 min & 05.09.2018 \\ \hline
Erstellung eines Meilensteinplans & 50 min & 05.09.2018 \\ \hline
Vorstellung des Meilensteinplans und Besprechung der anderen Ergebnisse & 40 min & 05.09.2018 \\ \hline
Aufarbeitung der Abgabetermine und Information der Gruppenteilnehmer  & 20 min & 10.09.2018 \\ \hline
Installation von Latex & 120 min & 10.09.2018 \\ \hline
Festlegung des Vorgehensmodells, durch Abwägung von Vor- und Nachteilen der verschiedenen Modelle (Ergebnis: Erweitertes Wasserfallmodell)
 & 40 min & 11.09.2018 \\ \hline
Teammeeting & 60 min & 22.09.2018 \\ \hline
Github Anmeldung und Einbindung des Projektes & 60 min & 13.10.2018 \\ \hline
Entwicklung des Layouts für das „Event“  & 70 min & 20.10.2018 \\ \hline
Entwicklung des Layouts für das „Note“ & 60 min & 21.10.2018 \\ \hline
Recherche über Listen, Checkboxen und Verwendung von mehreren Layouts & 50 min & 21.10.2018 \\ \hline
Besprechung der Layouts (Aufbau, etc.) & 50 min & 27.10.2018 \\ \hline
Entwicklung des Layouts für das „ToDo“ & 40 min & 27.10.2018 \\ \hline
Erstellung des Projekttagebuch mit Latex & 40 min & 30.10.2018 \\ \hline
Zusammentragung der momentanen Projektstände und Abschätzung, ob die Meilensteine erreicht werden können & 30 min & 17.11.2018 \\ \hline
Neuplanung eines Meilensteins und Abstimmung mit dem Team & 20 min & 26.11.2018 \\ \hline
Teammeeting & 50 min & 04.12.2018 \\ \hline
Koordination der anzufertigen Diagramme und Entwicklungsstand überprüfen& 10 min & 04.12.2018 \\ \hline
Entwicklung des Layouts für das „Image“ & 20 min & 12.12.2018 \\ \hline
Entwicklung des Layouts für die Overview &  60 min & 02.01.2019 \\ \hline
Aufteilung der Ausarbeitung & 20 min & 02.01.2019 \\ \hline
Entwicklung des Layouts für das Bedienleisten-Fragment & 20 min & 04.01.2019 \\ \hline
Meilenstein Umplanung & 10 min & 15.01.2019 \\ \hline
Erinnerung an die Erstellung der Kapitel der Ausarbeitung & 10 min & 23.01.2019 \\ \hline
Dokumentation & 240 min & 03.02.2019 \\ \hline
\end{longtable}
Summe in Minuten: 1270

\newpage
\subsubsection{Projekttagebuch Florian Rath}
\begin{longtable}{|p{10cm}|p{2cm}|p{2cm}|}
\hline
{\textbf{Beschreibung}} & {\textbf{Dauer}} & {\textbf{Datum}} \\ \hline

\end{longtable}

\newpage
\subsubsection{Projekttagebuch Robin Menzel}
\begin{longtable}{|p{10cm}|p{2cm}|p{2cm}|}
\hline
{\textbf{Beschreibung}} & {\textbf{Dauer}} & {\textbf{Datum}} \\ \hline

\end{longtable}

\newpage
\subsubsection{Projekttagebuch Ruthild Gilles}
\begin{longtable}{|p{10cm}|p{2cm}|p{2cm}|}
\hline
{\textbf{Beschreibung}} & {\textbf{Dauer}} & {\textbf{Datum}} \\ \hline
Aufgabe lesen und verstehen & 30 min & 05.09.2018 \\ \hline
Kick-Off Meeting zur Besprechung der Aufgabe, Verteilung der Rollen & 50 min & 05.09.2018 \\ \hline
Festlegung des Umfangs (Muss/Kann Kriterien) & 50 min & 05.09.2018 \\ \hline
Besprechung der in Aufgabenteilung entstandenen Ergebnisse & 40 min & 05.09.2018 \\ \hline
Installation von LateX & 120 min & 10.09.2018 \\ \hline
Projekttagebuch ausfüllen & 10 min & 11.09.2018 \\ \hline
Erstellung des Datenmodells & 50 min & 15.09.2018 \\ \hline
Teammeeting & 60 min & 22.09.2018 \\ \hline
Überlegung Aufgabenteilung, Aufgabenvergabe und Erklärung dieser bezogen auf die MainActivity & 40 min & 09.10.2018 \\ \hline
Installation und Einrichtung von GIT & 30 min & 13.10.2018 \\ \hline
Teammeeting Data-Team &  60 min & 26.10.2018 \\ \hline
Klonen des Data-Branches von GIT & 120 min & 27.10.2018 \\ \hline
Deklaration von gettern und settern für Datenobjekte & 120 min & 28.10.2018 \\ \hline
Projekttagebuch ausfüllen & 20 min & 31.10.2018 \\ \hline
Entwicklung von SetterService Klasse & 180 min & 16.11.2018 \\ \hline
Teammeeting Data-Team & 100 min & 20.11.2018 \\ \hline
Entwicklung von Service Klassen & 120 min & 21.11.2018 \\ \hline
Überlegung neuer Struktur für Services & 90 min & 22.11.2018 \\ \hline
Implementierung neuer Service-Struktur & 120 min & 30.11.2018 \\ \hline
Einbindung der Mockdaten & 80 min & 01.12.2018 \\ \hline
Implementierung weiterer Teile neuer Struktur & 90 min & 02.12.2018 \\ \hline
Änderungen an neuer Service-Struktur & 40 min & 03.12.2018 \\ \hline
Teammeeting & 50 min & 04.12.2018 \\ \hline
Entwicklung von Create-Klasse & 60 min & 04.12.2018 \\ \hline
Änderungen an Update-Klasse & 40 min & 17.12.2018 \\ \hline
Projekttagebuch ausfüllen & 20 min & 18.12.2018 \\ \hline
Bugfixing in Update- und Read-Klasse & 90 min & 02.01.2019 \\ \hline
Mockdaten durch Zugriff auf Datenbank ersetzt & 60 min & 02.01.2019 \\ \hline
Ergänzung einer Methode in Delete-Klasse & 30 min & 04.01.2019 \\ \hline
Neue Farben für GUI festlegen & 30 min & 07.01.2019 \\ \hline
Latex - Vorlage anpassen & 120 min & 14.01.2019 \\ \hline
Latex - Struktur für Ausarbeitung erstellen & 120 min & 18.01.2019 \\ \hline
Meinen Teil der Ausarbeitung schreiben & 180 min & 28.01.2019 \\ \hline
Latex - Meetingprotokolle einfügen & 180 min & 01.02.2019 \\ \hline
Latex - Quellcode einfügen & 120 min & 03.02.2019 \\ \hline
Latex - Ausarbeitungen der anderen einfügen & 180 min & 03.02.2019 \\ \hline
Projekttagebuch ausfüllen & 20 min & 03.02.2019 \\ \hline
Latex - Projekttagebücher einfügen & 120 min & 03.02.2019 \\ \hline
\end{longtable}
Summe in Minuten: 3090

\newpage
\subsubsection{Projekttagebuch Sertan Cetin}
\begin{longtable}{|p{10cm}|p{2cm}|p{2cm}|}
\hline
{\textbf{Beschreibung}} & {\textbf{Dauer}} & {\textbf{Datum}} \\ \hline

\end{longtable}

\newpage
\subsubsection{Projekttagebuch Yannick Rüttgers}
\begin{longtable}{|p{10cm}|p{2cm}|p{2cm}|}
\hline
{\textbf{Beschreibung}} & {\textbf{Dauer}} & {\textbf{Datum}} \\ \hline

\end{longtable}

\newpage
\subsection{Beschreibung von Problemen}
%%%%%%%%%%%
%Fabia
%%%%%%%%%%%

Im Lauf des Projektes konnten zwei Meilensteine nicht fristgerecht erfüllt werden. Einmal die Fertigstellung der Activities und die Fertigstellung der Dokumentation.
 
Die Fertigstellung der Activities und der Layouts war für den 01.12.2018 geplant, konnte jedoch nicht fristgerecht fertiggestellt werden, da projektfremde Tätigkeiten die Umsetzung verzögerten. Beispielsweise schrieben wir eine Klausur, die bei der Projektplanung noch nicht bekannt war. Als Ergebnis wurde der Meilenstein auf ein späteres Datum geplant und konnte danach fristgerecht erfüllt werden. Diese Verschiebung hatte jedoch keinen negativen Einfluss auf die Fertigstellung des Projektes.

Auch der Meilenstein „Dokumentation fertig“ konnte nicht fristgerecht zum 01.02.2019 erreicht werden. Auch in diesem Fall waren projektfremde Tätigkeiten die Ursache für den Verzug. Jedoch musste der Meilenstein nur um 3 Tage verschoben werden, wodurch das Projektergebnis nicht gefährdet wurde.

Weitere Probleme oder Verzögerungen traten nicht auf und das Projekt konnte erfolgreich am 09.02.2019 vorgestellt werden.



