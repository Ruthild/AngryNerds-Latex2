%!TEX root = ../Thesis.tex
\section{Beschreibung des Projektverlaufs}
\fancyhead[R]{Beschreibung des Projektverlaufs}
\label{instal}

\subsection{Tatsächliche Aufgabenverteilung im Team (Fabia Schmid)}
%%%%%%%%%%%
%Fabia
%%%%%%%%%%%

\begin{longtable}{|p{4cm}|p{10cm}|}
\hline
{\textbf{Name}} & {\textbf{Aufgaben}}  \\ \hline
Ruthild Gilles & Erstellung Service-Klassen,

Schreiben eines Protokolls,

Latex-Beauftragte,

Projekttagebuch führen, 

Dokumentation anfertigen \\ \hline 
Fabia Schmid  & Projektsteuerung und -planung, 

Erstellung der Layouts für die ActivityOverview,  

Erstellung der OnClickListener für die ActivityOverview,  

Projekttagebuch führen, 

Dokumentation anfertigen \\ \hline

Jan Beilfuß & Datenbankzugriffe, 

Start-Up-Lade-Routine von Note, 

Bilderhandling in Note und generell, 

Note-Applicationlogicstrukturierung, 

Unterstützung im Umgang mit Git, 

Mockdatenerstellung,

Projekttagebuch führen, 

Dokumentation anfertigen \\ \hline
Yannick Rüttgers & Erstellung ActivityOverview,  

Planung der Navigation zwischen Klassen, 

Erster Latexentwurfprojekttagebuch, 

Latex-Beauftragter, 

Projekttagebuch führen, 

Dokumentation anfertigen \\ \hline
Robin Menzel & Zusammenführung des Quellcodes, 

Administration der Versionsverwaltung,

Planung der Layouts,

Hilfe bei der Aufteilung von den Activities,

Erstellung des Mockups, 

Erstellung der Event Activity, 

Projekttagebuch führen, 

Dokumentation anfertigen \\ \hline
Florian Rath  & Aufteilung der Activities, 

Erstellung Activity ToDo, 

Projekttagebuch führen, 

Dokumentation anfertigen \\ \hline
Joscha Nassenstein & Erstellung von Note, 

Erstellung der TEN-Klassen,  

Erstellung Datendiagramm, 

Projekttagebuch führen, 

Dokumentation anfertigen \\ \hline
Sertan Cetin &  Aufteilung der Activities, 

Erstellung Activity ToDo, 

Projekttagebuch führen, 

Dokumentation anfertigen \\ \hline
\end{longtable}

\subsection{Teammeetingprotokolle}
%%%%%%%%%%%
%Ruthild
%%%%%%%%%%%

Hier den Text einfach hin kopieren.

\subsection{Projekttagebücher aller Teammitglieder}
%%%%%%%%%%%
%Ruthild
%%%%%%%%%%%
\subsubsection{Projekttagebuch Jan Beilfuß}
\begin{longtable}{|p{10cm}|p{2cm}|p{2cm}|}
\hline
{\textbf{Beschreibung}} & {\textbf{Dauer}} & {\textbf{Datum}} \\ \hline

\end{longtable}

\newpage
\subsubsection{Projekttagebuch Joscha Nassenstein}
\begin{longtable}{|p{10cm}|p{2cm}|p{2cm}|}
\hline
{\textbf{Beschreibung}} & {\textbf{Dauer}} & {\textbf{Datum}} \\ \hline

\end{longtable}

\newpage
\subsubsection{Projekttagebuch Fabia Schmid}
\begin{longtable}{|p{10cm}|p{2cm}|p{2cm}|}
\hline
{\textbf{Beschreibung}} & {\textbf{Dauer}} & {\textbf{Datum}} \\ \hline
Aufgabe lesen und verstehen & 30 min & 05.09.2018 \\ \hline
Kick-Off Meeting zur Besprechung der Aufgabe, Verteilung der Rollen & 50 min & 05.09.2018 \\ \hline
Erstellung eines Meilensteinplans & 50 min & 05.09.2018 \\ \hline
Vorstellung des Meilensteinplans und Besprechung der anderen Ergebnisse & 40 min & 05.09.2018 \\ \hline
Aufarbeitung der Abgabetermine und Information der Gruppenteilnehmer  & 20 min & 10.09.2018 \\ \hline
Installation von Latex & 120 min & 10.09.2018 \\ \hline
Festlegung des Vorgehensmodells, durch Abwägung von Vor- und Nachteilen der verschiedenen Modelle (Ergebnis: Erweitertes Wasserfallmodell)
 & 40 min & 11.09.2018 \\ \hline
Teammeeting & 60 min & 22.09.2018 \\ \hline
Github Anmeldung und Einbindung des Projektes & 60 min & 13.10.2018 \\ \hline
Entwicklung des Layouts für das „Event“  & 70 min & 20.10.2018 \\ \hline
Entwicklung des Layouts für das „Note“ & 60 min & 21.10.2018 \\ \hline
Recherche über Listen, Checkboxen und Verwendung von mehreren Layouts & 50 min & 21.10.2018 \\ \hline
Besprechung der Layouts (Aufbau, etc.) & 50 min & 27.10.2018 \\ \hline
Entwicklung des Layouts für das „ToDo“ & 40 min & 27.10.2018 \\ \hline
Erstellung des Projekttagebuch mit Latex & 40 min & 30.10.2018 \\ \hline
Zusammentragung der momentanen Projektstände und Abschätzung, ob die Meilensteine erreicht werden können & 30 min & 17.11.2018 \\ \hline
Neuplanung eines Meilensteins und Abstimmung mit dem Team & 20 min & 26.11.2018 \\ \hline
Teammeeting & 50 min & 04.12.2018 \\ \hline
Koordination der anzufertigen Diagramme und Entwicklungsstand überprüfen& 10 min & 04.12.2018 \\ \hline
Entwicklung des Layouts für das „Image“ & 20 min & 12.12.2018 \\ \hline
Entwicklung des Layouts für die Overview &  60 min & 02.01.2019 \\ \hline
Aufteilung der Ausarbeitung & 20 min & 02.01.2019 \\ \hline
Entwicklung des Layouts für das Bedienleisten-Fragment & 20 min & 04.01.2019 \\ \hline
Meilenstein Umplanung & 10 min & 15.01.2019 \\ \hline
Erinnerung an die Erstellung der Kapitel der Ausarbeitung & 10 min & 23.01.2019 \\ \hline
Dokumentation & 240 min & 03.02.2019 \\ \hline
\end{longtable}
Summe in Minuten: 1270

\newpage
\subsubsection{Projekttagebuch Florian Rath}
\begin{longtable}{|p{10cm}|p{2cm}|p{2cm}|}
\hline
{\textbf{Beschreibung}} & {\textbf{Dauer}} & {\textbf{Datum}} \\ \hline

\end{longtable}

\newpage
\subsubsection{Projekttagebuch Robin Menzel}
\begin{longtable}{|p{10cm}|p{2cm}|p{2cm}|}
\hline
{\textbf{Beschreibung}} & {\textbf{Dauer}} & {\textbf{Datum}} \\ \hline
Aufgabe lesen und verstehen & 30 min & 05.09.2018 \\ \hline
Kick-Off Meeting zur Besprechung der Aufgabe, Verteilung der Rollen & 50 min & 05.09.2018 \\ \hline
Konzeption eines Mockups und den Activities & 50 min & 05.09.2018 \\ \hline
Besprechung der in Aufgabenteilung entstandenen Ergebnisse & 40 min & 05.09.2018 \\ \hline
Installation von Adobe XD für MockUps & 20 min & 05.09.2018 \\ \hline
Installation und Einrichtung von Git & 20 min & 05.09.2018 \\ \hline
Erstellung und Einrichtung eines GitHub Repositorys & 40 min & 05.09.2018 \\ \hline
Erstellung von MockUps in der ersten Version & 180 min & 06.09.2018 \\ \hline
Wahl der LateX Distribution & 20 min & 10.09.2018 \\ \hline
Installation von LateX & 120 min & 10.09.2018 \\ \hline
Projekttagebuch pflegen & 10 min & 11.09.2018\\ \hline
Weiterentwicklung des MockUps & 30 min & 15.09.2018\\ \hline
Meeting zur Besprechung der Ergebnisse aus dem Design- und Daten-Team & 60 min & 22.09.2018\\ \hline
Korrekturen am MockUp und Teilen der Ergebnisse im Microsoft Teams & 30 min & 23.09.2018\\ \hline
Kommunikation mit Data Team im Bezug auf Objekte und Übergabe an Activities & 30 min & 23.10.2018 \\ \hline
Erstellung des XML-Layouts der Event-Activity & 180 min & 27.10.2018\\ \hline
Recherche und Design von Farben für die Hintergründe von Activities und Erstellung von res-Files & 60 min & 28.10.2018\\ \hline
Probleme im Git Repository beheben und einrichten von Branches & 30 min & 28.10.2018\\ \hline
Erstellung des Projekttagebuchs in Latex & 40 min & 30.10.2018 \\ \hline
Weiterentwicklung des XML-Layouts der Event-Activity & 120 min & 15.10.2018\\ \hline
Recherche Umsetzung der Datumsauswahl (DatePicker) & 60 min & 15.10.2018\\ \hline
Git Repository neu aufsetzen (Merger und neu anlegen) & 180 min & 01.12.2018\\ \hline
Teammeeting & 50 min & 04.12.2018\\ \hline
Erstellung Notwendiger Klassen für die Event Activity & 600 min & 10.12.2018 \\ \hline
Recherche nach App oder Toolbar für API 19 & 120 min & 12.12.2018 \\ \hline
Implementation einer Toolbar für alle TENs & 260 min & 13.12.2018 \\ \hline
Implementation des 3-Punkt-Menüs in der Toolbar & 30 min & 13.12.2018 \\ \hline
Implementation des Öffnens der Activity (Öffnen mit ID, ohne, Übergänge) & 90 min &  17.12.2018\\ \hline
Erstellung des Projekttagebuchs in Latex & 15 min & 19.12.2018 \\ \hline
Recherche DialogPicker um Zeit und Datum für Events auszuwählen & 120 min & 23.12.2018 \\ \hline
Implementation von Date- und Time-Picker & 300 min & 27.12.2018 \\ \hline
Implementation der dynamischen Reminder in der GUI und dem Auswahldialog & 390 min & 29.12.2018 \\ \hline
Recherche Notification Service und Alarm Manager & 90 min & 30.12.2019 \\ \hline
Implementation von Remindern inkl. Notification Service und Alarm Manager & 120 min & 31.01.2019 \\ \hline
Ausführliches Testen 05.01.2019 & 120 min & 04.01.2019\\ \hline
Anpassungen auf API 19 & 120 min & 05.01.2019 \\ \hline
Ausführliches Testen 09.01.2019 & 90 min & 10.01.2019\\ \hline
Anpassungen an dem Alarm Manager um Reminder vor aktueller Zeit zu ignorieren & 30 min & 10.01.2019 \\ \hline
Recherche nach Schnittstellen zu Google Maps & 15 min & 10.01.2019 \\ \hline
Implementation eines Buttons um die Adresse eines Events in Google Maps zu öffnen & 30 min 6 & 10.01.2019 \\ \hline
GUI Anpassungen für den Navigations-Button & 30 min & 11.01.2019 \\ \hline
Implementation einer Teilen-Funktionalität für Text-basiertes Event & 45 min & 13.01.2019 \\ \hline
Implementation einer Export-Funktion in andere Kalender Applikationen & 30 min & 13.01.2019 \\ \hline
Implementation der Wiederholungsfunktion (Einmalig, Täglich, ...) & 180 min & 17.01.2019 \\ \hline
Erstellung der Dokumentation 1 & 120 min & 28.01.2019 \\ \hline
Erstellung des Projekttagebuchs in Latex & 60 min & 29.12.2018 \\ \hline
Erstellung der Dokumentation 2 & 140 min & 30.01.2019 \\ \hline
\end{longtable}
Summe in Minuten: 4595

\newpage
\subsubsection{Projekttagebuch Ruthild Gilles}
\begin{longtable}{|p{10cm}|p{2cm}|p{2cm}|}
\hline
{\textbf{Beschreibung}} & {\textbf{Dauer}} & {\textbf{Datum}} \\ \hline
Aufgabe lesen und verstehen & 30 min & 05.09.2018 \\ \hline
Kick-Off Meeting zur Besprechung der Aufgabe, Verteilung der Rollen & 50 min & 05.09.2018 \\ \hline
Festlegung des Umfangs (Muss/Kann Kriterien) & 50 min & 05.09.2018 \\ \hline
Besprechung der in Aufgabenteilung entstandenen Ergebnisse & 40 min & 05.09.2018 \\ \hline
Installation von LateX & 120 min & 10.09.2018 \\ \hline
Projekttagebuch ausfüllen & 10 min & 11.09.2018 \\ \hline
Erstellung des Datenmodells & 50 min & 15.09.2018 \\ \hline
Teammeeting & 60 min & 22.09.2018 \\ \hline
Überlegung Aufgabenteilung, Aufgabenvergabe und Erklärung dieser bezogen auf die MainActivity & 40 min & 09.10.2018 \\ \hline
Installation und Einrichtung von GIT & 30 min & 13.10.2018 \\ \hline
Teammeeting Data-Team &  60 min & 26.10.2018 \\ \hline
Klonen des Data-Branches von GIT & 120 min & 27.10.2018 \\ \hline
Deklaration von gettern und settern für Datenobjekte & 120 min & 28.10.2018 \\ \hline
Projekttagebuch ausfüllen & 20 min & 31.10.2018 \\ \hline
Entwicklung von SetterService Klasse & 180 min & 16.11.2018 \\ \hline
Teammeeting Data-Team & 100 min & 20.11.2018 \\ \hline
Entwicklung von Service Klassen & 120 min & 21.11.2018 \\ \hline
Überlegung neuer Struktur für Services & 90 min & 22.11.2018 \\ \hline
Implementierung neuer Service-Struktur & 120 min & 30.11.2018 \\ \hline
Einbindung der Mockdaten & 80 min & 01.12.2018 \\ \hline
Implementierung weiterer Teile neuer Struktur & 90 min & 02.12.2018 \\ \hline
Änderungen an neuer Service-Struktur & 40 min & 03.12.2018 \\ \hline
Teammeeting & 50 min & 04.12.2018 \\ \hline
Entwicklung von Create-Klasse & 60 min & 04.12.2018 \\ \hline
Änderungen an Update-Klasse & 40 min & 17.12.2018 \\ \hline
Projekttagebuch ausfüllen & 20 min & 18.12.2018 \\ \hline
Bugfixing in Update- und Read-Klasse & 90 min & 02.01.2019 \\ \hline
Mockdaten durch Zugriff auf Datenbank ersetzt & 60 min & 02.01.2019 \\ \hline
Ergänzung einer Methode in Delete-Klasse & 30 min & 04.01.2019 \\ \hline
Neue Farben für GUI festlegen & 30 min & 07.01.2019 \\ \hline
Latex - Vorlage anpassen & 120 min & 14.01.2019 \\ \hline
Latex - Struktur für Ausarbeitung erstellen & 120 min & 18.01.2019 \\ \hline
Meinen Teil der Ausarbeitung schreiben & 180 min & 28.01.2019 \\ \hline
Latex - Meetingprotokolle einfügen & 180 min & 01.02.2019 \\ \hline
Latex - Quellcode einfügen & 120 min & 03.02.2019 \\ \hline
Latex - Ausarbeitungen der anderen einfügen & 180 min & 03.02.2019 \\ \hline
Projekttagebuch ausfüllen & 20 min & 03.02.2019 \\ \hline
Latex - Projekttagebücher einfügen & 120 min & 03.02.2019 \\ \hline
\end{longtable}
Summe in Minuten: 3090

\newpage
\subsubsection{Projekttagebuch Sertan Cetin}
\begin{longtable}{|p{10cm}|p{2cm}|p{2cm}|}
\hline
{\textbf{Beschreibung}} & {\textbf{Dauer}} & {\textbf{Datum}} \\ \hline

\end{longtable}

\newpage
\subsubsection{Projekttagebuch Yannick Rüttgers}
\begin{longtable}{|p{10cm}|p{2cm}|p{2cm}|}
\hline
{\textbf{Beschreibung}} & {\textbf{Dauer}} & {\textbf{Datum}} \\ \hline

\end{longtable}

\newpage
\subsection{Beschreibung von Problemen (Fabia Schmid)}
%%%%%%%%%%%
%Fabia
%%%%%%%%%%%

Im Lauf des Projektes konnten zwei Meilensteine nicht fristgerecht erfüllt werden. Einmal die Fertigstellung der Activities und die Fertigstellung der Dokumentation.
 
Die Fertigstellung der Activities und der Layouts war für den 01.12.2018 geplant, konnte jedoch nicht fristgerecht fertiggestellt werden, da projektfremde Tätigkeiten die Umsetzung verzögerten. Beispielsweise schrieben wir eine Klausur, die bei der Projektplanung noch nicht bekannt war. Als Ergebnis wurde der Meilenstein auf ein späteres Datum geplant und konnte danach fristgerecht erfüllt werden. Diese Verschiebung hatte jedoch keinen negativen Einfluss auf die Fertigstellung des Projektes.

Auch der Meilenstein „Dokumentation fertig“ konnte nicht fristgerecht zum 01.02.2019 erreicht werden. Auch in diesem Fall waren projektfremde Tätigkeiten die Ursache für den Verzug. Jedoch musste der Meilenstein nur um 3 Tage verschoben werden, wodurch das Projektergebnis nicht gefährdet wurde.

Weitere Probleme oder Verzögerungen traten nicht auf und das Projekt konnte erfolgreich am 09.02.2019 vorgestellt werden.



