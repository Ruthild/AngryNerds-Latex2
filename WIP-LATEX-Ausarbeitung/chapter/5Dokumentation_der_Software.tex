%!TEX root = ../Thesis.tex
\section{Dokumentation der Software}
\fancyhead[R]{Dokumentation der Software}
\label{instal}

\subsection{Dokumentation der Paketstruktur (Sertan Cetin)}
%%%%%%%%%%%
%Sertan
%%%%%%%%%%%
 
\subsection{Dokumentation der Activities}
%%%%%%%%%%%
%Alle
%%%%%%%%%%%

Hier den Text einfach hin kopieren.

\subsection{Dokumentation der Navigation zwischen Activities (Yannick Rüttgers)}
%%%%%%%%%%%
%Yannick
%%%%%%%%%%%

Hier den Text einfach hin kopieren.

\subsection{Dokumentation der Activity-übergreifenden, persistenten Datenhaltung (Jan Beilfuß)}
%%%%%%%%%%%
%Jan
%%%%%%%%%%%

Hier den Text einfach hin kopieren.

\newpage
\subsection{Dokumentation der Activity-übergreifenden Klassen (Ruthild Gilles)}
%%%%%%%%%%%
%Ruthild
%%%%%%%%%%%

Zu den Activity-übergreifenden Klassen gehören abgesehen von den Query-Klassen für die persistente Datenhaltung auch die Service-Klassen und die Models-Klassen. Die Models-Klassen enthalten die TEN-Klasse und die einzelnen Todo-, Event- und Note-Klassen. Hier sind auch weitere Util-Klassen untergeordnet.

Damit die Activities die Daten der TEN-Objekte auf der dokumentenbasierten Datenbank speichern können, wurden Service Klassen implementiert. Hierzu gehörten hauptsächlich Klassen zum Erzeugen neuer TEN-Objekte (Create), zum Erhalten bereits gespeicherter TEN-Objekte von der Datenbank (Read), zum Speichern veränderter TEN-Objekte (Update) und zum Löschen von TEN-Objekten von der Datenbank (Delete). Diese sogenannte CRUD-Operationen wurden in den entsprechenden Klassen teilweise für alle drei verschiedenen Objekttypen einzeln eingefügt, teilweise aber auch für TEN-Objekte im Allgemeinen. Dank Polymorphie können die jeweils einzelnen Objekttypen ebenfalls an die entsprechenden Methoden übergeben werden.

Diese Struktur wurde während der Planungsphase vom Datenteam überlegt und während der Implementierung angepasst. Wie auch schon in der Planungsphase definiert enthält die Create-Klasse Methoden, die ein neues leeres Todo, Event oder Note zurückgeben. Diese Methode ruft den Konstruktor der jeweiligen TEN-Klasse auf. Eine Interaktion mit der Datenbank ist hier noch nicht nötig.

Die Read-Klasse hingegen muss auf die Datenbank zugreifen, um entweder alle TEN-Objekte an die Main-Activity in einem ListArray zu übergeben oder aber ein spezielles Todo, Event oder Note, welches von den einzelnen Todo-, Event- oder Note-Activities aufgerufen werden kann. Damit das gewünschte TEN-Objekt in der Datenbank gefunden werden kann, benötigen die Read-Methoden die ID des gewünschten Objektes. Dieses wird als String beim Aufruf der jeweiligen Methode übergeben.

Die Update-Klasse dient zum Speichern von Änderungen an Todo-, Event- und Note-Objekten. Dazu überprüft die Methode, der ein TEN-Objekt übergeben wurde, ob dieses bereits auf der Datenbank existiert. Ist dies der Fall, wird eine Methode zum Ausführen des Update-Befehls auf der Datenbank aufgerufen. Ist dies nicht der Fall, wird eine Methode zum Ausführen des Insert-Befehls auf der Datenbank aufgerufen. Da die Todo-, Event- und Note-Klassen von der TEN-Klasse erben, kann hier Polymorphie angewandt werden. Es ist nur eine Methode zum Speichern notwendig.

Für das Löschen von TEN-Objekten sind in der Delete-Klasse zwei Methode vorhanden. Die eine löscht nur ein übergebenes TEN-Objekt aus der Datenbank, indem es eine entsprechende Methode aus einer der Repository-Klassen aufruft und dieser die ID des TEN-Objektes übergibt. Sollen jedoch mehrere TEN-Objekte auf einmal gelöscht werden, kann der zweiten Methode eine Array List, die mehrere zu löschende Objekte enthält, übergeben werden. Dabei müssen die Objekte nicht alle von dem gleichen Datentyp sein, sondern können Todo-, Event- und Note-Objekte enthalten. Die Methode iteriert durch die übergebene Array List und ruft für jeden Eintrag die Methode zum Löschen eines Objektes auf.
 
%%%%%%%%%%%%%%%%%%%%%%%%%%%%%%%%%%%%%%%%%%%%%%%%%%%%%
%Als Beispiel
%Bitte noch entfernen:

\subsection{Kapitel mit Abbildung}

So kann man Abbildungen einfügen:

\begin{figure}[H]
\centering
\begin{minipage}[t]{1\textwidth} % Breite, z.B. 1\textwidth		
\caption{Abbildungsbeschriftung} % Überschrift
\includegraphics[width=1\textwidth]{img/fhdw}\\ % Pfad
\source{Screenshot aus der Benutzeroberfläche} % Quelle
\end{minipage}
\end{figure}

