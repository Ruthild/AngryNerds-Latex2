%!TEX root = ../Thesis.tex
\section{Dokumentation der Software}
\fancyhead[R]{Dokumentation der Software}
\label{instal}

\subsection{Dokumentation der Paketstruktur (Sertan Cetin)}
%%%%%%%%%%%
%Sertan
%%%%%%%%%%%
 
\subsection{Dokumentation der Activities}
%%%%%%%%%%%
%Alle
%%%%%%%%%%%

Hier den Text einfach hin kopieren.

\subsection{Dokumentation der Navigation zwischen Activities (Yannick Rüttgers)}
%%%%%%%%%%%
%Yannick
%%%%%%%%%%%

Hier den Text einfach hin kopieren.

\subsection{Dokumentation der Activity-übergreifenden, persistenten Datenhaltung (Jan Beilfuß)}
%%%%%%%%%%%
%Jan
%%%%%%%%%%%

Hier den Text einfach hin kopieren.

\newpage
\subsection{Dokumentation der Activity-übergreifenden Klassen (Ruthild Gilles)}
%%%%%%%%%%%
%Ruthild
%%%%%%%%%%%

Zu den Activity-übergreifenden Klassen gehören abgesehen von den Query-Klassen für die persistente Datenhaltung auch die Service-Klassen und die Models-Klassen. Die Models-Klassen enthalten die TEN-Klasse und die einzelnen Todo-, Event- und Note-Klassen. Hier sind auch weitere Util-Klassen untergeordnet.

Damit die Activities die Daten der TEN-Objekte auf der dokumentenbasierten Datenbank speichern können, wurden Service Klassen implementiert. Hierzu gehörten hauptsächlich Klassen zum Erzeugen neuer TEN-Objekte (Create), zum Erhalten bereits gespeicherter TEN-Objekte von der Datenbank (Read), zum Speichern veränderter TEN-Objekte (Update) und zum Löschen von TEN-Objekten von der Datenbank (Delete). Diese sogennante CRUD-Operationen wurden in den entsprechenden Klassen teilweise für alle drei verschiedenen Objekttypen einzeln eingefügt, teilweise aber auch für TEN-Objekte im Allgemeinen. Dank Polymorphie können die jeweils einzelnen Objekttypen ebenfalls an die entsprechenden Methoden übergeben werden.
 
%%%%%%%%%%%%%%%%%%%%%%%%%%%%%%%%%%%%%%%%%%%%%%%%%%%%%
%Als Beispiel
%Bitte noch entfernen:

\subsection{Kapitel mit Abbildung}

So kann man Abbildungen einfügen:

\begin{figure}[H]
\centering
\begin{minipage}[t]{1\textwidth} % Breite, z.B. 1\textwidth		
\caption{Abbildungsbeschriftung} % Überschrift
\includegraphics[width=1\textwidth]{img/fhdw}\\ % Pfad
\source{Screenshot aus der Benutzeroberfläche} % Quelle
\end{minipage}
\end{figure}

